\chapter{Dec.~2 --- Distributions, Part 4}

\section{More on Convolutions of Distributions}

\begin{remark}
  Recall that if $\nu \in \mathcal{D}'(\R)$
  and $f, g \in C^\infty_c(\R)$, then we
  define
  \[
    \langle f, g * \nu \rangle
    := \langle f * \widetilde{g}, \nu \rangle.
  \]
  We want to make a similar
  definition for $\mu, \nu \in \mathcal{D}'(\R)$:
  \[
    \langle f, \mu * \nu \rangle
    := \langle f * \widetilde{\mu}, \nu \rangle,
    \quad f \in C^\infty_c(\R).
  \]
  Note that we only know
  $f * \mu \in C^\infty(\R)$,
  not necessarily $C^\infty_c(\R)$. We know
  \[
    \supp(\mu)
    = \bigcap \{\text{closed sets $F \subseteq \R$} : \text{$\mu = 0$ off $F$}\}.
  \]
\end{remark}

\begin{theorem}
  $\mathcal{E}'(\R) = C^\infty(\R)^* = \{\mu \in \mathcal{D}'(\R) : \mu \text{ is compactly supported}\}$.
\end{theorem}

\begin{definition}
  If $\mu, \nu \in \mathcal{D}'(\R)$
  and at least one is compactly supported,
  then we can define
  \[
    \langle f, \mu * \nu \rangle
    := \langle f * \widetilde{\mu}, \nu \rangle,
    \quad f \in C^\infty_c(\R).
  \]
\end{definition}

\begin{exercise}
  Show that if at least
  one of $\nu$ or $\mu$  is compactly
  supported, then
  \[
    f * (\mu * v) = (f * \mu) * \nu,
    \quad f \in C^\infty_c(\R).
  \]
\end{exercise}

\begin{exercise}
  Show that
  $(H * \delta') * 1 \ne H * (\delta' * 1)$,
  where $H$ is the Heaviside function,
  $\delta$ is the Dirac delta, and
  $1$ is the constant function.
\end{exercise}

\section{More on Distributional Derivatives}

\begin{lemma}
  Let $g \in C(\R)$ such that
  $Dg \in C(\R)$. Then
  $g \in C^1(\R)$ and
  $Dg = g'$.
\end{lemma}

\begin{proof}
  We give the idea of the proof. Let
  $h \in \R$ and suppose
  $f \in C^\infty_c(\R)$. Then
  \[
    \int_0^h (\widetilde{f} * g)'(t)\, dt
    = (\widetilde{f} * g)(h)
    - (\widetilde{f} * g)(0)
  \]
  by the fundamental theorem of calculus.
  Check as an exercise that
  $(\widetilde{f} * g)'(t) = (\widetilde{f} * Dg)(t)$, then
  \[
    \int \overline{f(x)} g(x)\, dx
    = \overline{\langle f, g \rangle}
    = (\widetilde{f} * g)(h)
    - \int_0^h (\widetilde{f} * Dg)(t)\, dt.
  \]
  One can continue to show that
  $Dg$ and $g'$ agree.
\end{proof}

\pagebreak

\begin{exercise}
  Show that if
  $\mu \in \mathcal{D}'(\R)$ and
  $f \in C^\infty_c(\R)$, then
  \[
    \overline{\langle f, \mu \rangle}
    = (\widetilde{f} * \mu)(t)
    - \int_0^h (\widetilde{f} * D\mu)(t)\, dt.
  \]
\end{exercise}

\begin{theorem}
  If $\mu \in \mathcal{D}'(\R)$, then
  $D\mu \in C(\R)$ (i.e. there exists
  $g \in C(\R)$ such that
  $\langle f, g \rangle = \langle f, D\mu \rangle = -\langle f', \mu \rangle$)
  if and only if $\mu \in C^1(\R)$.
\end{theorem}

\begin{theorem}\label{thm:distribution_derivative_continuous}
  Fix $\mu \in \mathcal{D}'(\R)$. Then
  for each compact $K \subseteq \R$,
  there exists $g \in C(\R)$ and
  $k \ge 0$ such that
  $\langle f, \mu \rangle = \langle f, D^k g \rangle$
  for all $f \in C^\infty(K)$.
\end{theorem}

\begin{remark}
  Theorem \ref{thm:distribution_derivative_continuous}
  says that in some sense,
  distributions are exactly
  the derivatives of functions.
\end{remark}

\section{Distributional Fourier Transform}

\begin{remark}
  Recall that the Fourier transform
  is unitary on $L^2(\R)$, i.e.
  \[
    \langle f, g \rangle
    = \int_{-\infty}^\infty f(x) \overline{g(x)}\, dx
    = \int_{-\infty}^\infty \widehat{f}(\xi) \overline{\widehat{g}(\xi)}\, d\xi
    = \langle \widehat{f}, \widehat{g} \rangle.
  \]
  Also recall that if
  $f \in \mathcal{S}(\R)$, then
  $\widehat{f} \in \mathcal{S}(\R)$ also
  (this is not the case
  with $C^\infty_c(\R)$).
\end{remark}

\begin{definition}
  The \emph{Fourier transform} of a
  tempered distribution
  $\mu \in \mathcal{S}'(\R)$ is
  implicitly defined by
  \[
    \langle \widehat{f}, \widehat{\mu} \rangle
    := \langle f, \mu \rangle,
    \quad f \in \mathcal{S}(\R).
  \]
  This defines $\widehat{\mu}$
  on all of $\mathcal{S}(\R)$ since
  $\widehat{f}$ can be any element
  of $\mathcal{S}(\R)$. More explicitly,
  \[
    \langle f, \widehat{\mu} \rangle
    = \langle \widecheck{f}, \mu \rangle,
    \quad f \in \mathcal{S}(\R).
  \]
\end{definition}

\begin{lemma}
  If $\mu \in \mathcal{S}'(\R)$, then
  $\widehat{\mu} \in \mathcal{S}'(\R)$.
\end{lemma}

\begin{proof}
  Since $\mu \in \mathcal{S}'(\R)$, there
  exist $C, M, N \ge 0$ such that
  \[
    |\langle f, \mu \rangle|
    \le C \sum_{m = 0}^M \sum_{n = 0}^N
    \|x^m f^{(n)}\|_\infty.
  \]
  Then we can compute
  \[
    |\langle f, \widehat{\mu} \rangle|
    = |\langle \widecheck{f}, \mu \rangle|
    \le C \sum_{m = 0}^M \sum_{n = 0}^N
    \|x^m \widecheck{f}^{(n)}\|_\infty.
  \]
  For $m = n = 0$, we have
  $\|\widecheck{f}\|_\infty \le \|f\|_1 \le C'(\|f\|_\infty + \|x^2 f(x)\|_\infty)$
  (split into $|x| \le 1$ and $|x| > 1$).
  We can apply a similar argument to the
  other $m, n$ to bound the
  remaining terms.
\end{proof}

\begin{example}
  We compute $\widehat{\delta}$.
  Let $f \in \mathcal{S}(\R)$. Then
  \[
    \langle \widehat{f}, \widehat{\delta} \rangle
    = \langle f, \delta \rangle
    = f(0)
    = (\widehat{f})^\vee(0)
    = \int_{-\infty}^\infty \widehat{f}(\xi)\, d\xi
    = \langle \widehat{f}, 1 \rangle.
  \]
  This is true for every $f \in \mathcal{S}(\R)$,
  hence $\widehat{\delta} = 1$.
\end{example}

\begin{example}
  We compute $\widehat{1}$.
  For $f \in \mathcal{S}(\R)$, we have
  \[
    \langle \widehat{f}, \widehat{1} \rangle
    = \langle f, 1 \rangle
    = \int f(x)\, dx
    = \widehat{f}(0)
    = \langle \widehat{f}, \delta \rangle.
  \]
  Thus we see that
  $\widehat{1} = \delta$.
\end{example}

\begin{example}
  Formally, we can write
  \[
    \widehat{\delta}(\xi)
    = \int_{-\infty}^\infty \delta(x)
    e^{-2\pi i \xi x}\, dx
    = \int_{-\infty}^\infty e^{-2\pi i \xi x}\, d\delta(x)
    = e^{-2\pi i \xi \cdot 0}
    = 1.
  \]
  This somewhat makes sense, with some
  hand-waving.
  On the other hand
  \[
    \widehat{1}(\xi)
    = \int_{-\infty}^\infty 1 \cdot e^{-2\pi i \xi x}\, dx
    = \delta(\xi)
  \]
  does not really make sense at all.
\end{example}

\begin{exercise}
  Show that $\mu \mapsto \widehat{\mu}$
  is a continuous
  bijection $\mathcal{S}'(\R) \to \mathcal{S}'(\R)$.
\end{exercise}

\begin{exercise}
  Show that
  if $1 \le p \le 2$ and
  $f \in L^p(\R)$, then the
  usual Fourier transform agrees
  with the distributional Fourier transform
  (if we interpret the usual
  $\widehat{f} \in L^{p'}(\R) \subseteq \mathcal{S}'(\R)$).
\end{exercise}

\begin{remark}
  If $2 < p \le \infty$, then we still have
  $L^{p}(\R) \subseteq \mathcal{S}'(\R)$,
  so every $f \in L^p(\R)$ has a
  distributional Fourier transform
  $\widehat{f} \in \mathcal{S}'(\R)$, but
  in general it cannot be identified with
  a function.
\end{remark}

\begin{remark}
  $\mathcal{F} L^\infty(\R)$ is called
  the space of \emph{pseudo-measures}.
\end{remark}

\begin{exercise}
  Show that if $\mu \in \mathcal{S}'(\R)$
  and $\theta \in \mathcal{S}(\R)$, then
  \[
    (\theta \mu)^\wedge
    = \widehat{\theta} * \widehat{\mu}
    \quad \text{and} \quad
    (\theta * \mu)^\wedge
    = \widehat{\theta} \widehat{\mu}.
  \]
  Moreover, $(D\mu)^\wedge = 2\pi i \xi \widehat{\mu}$
  and $D\widehat{\mu} = (-2\pi i x \mu)^\wedge$.
\end{exercise}

\begin{theorem}
  $(\sum_{n = -\infty}^\infty \delta_n)^\wedge = \sum_{n = -\infty}^\infty \delta_n$.
\end{theorem}

\begin{proof}
  We can compute that
  \[
    \left\langle \widehat{f}, \Big(\sum_{n = -\infty}^\infty \delta_n\Big)^\wedge \right\rangle
    = \left\langle f, \sum_{n = -\infty}^\infty \delta_n \right\rangle
    = \sum_{n = -\infty}^\infty f(n)
    = \sum_{k = -\infty}^\infty \widehat{f}(k)
    = \left\langle \widehat{f}, \sum_{n = -\infty}^\infty \delta_n \right\rangle,
  \]
  where we used the Poisson summation
  formula in the third equality.
\end{proof}
