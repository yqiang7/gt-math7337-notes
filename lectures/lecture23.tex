\chapter{Nov.~20 --- Distributions, Part 2}

\section{Convergence with Families of Seminorms}

\begin{example}
  Like $\mathcal{S}(\R)$, many
  other spaces are defined by a
  family of seminorms. Recall that
  \[
    L^1_\mathrm{loc}(\R)
    = \{
      \text{measurable } f : f \text{ is integrable on compact } K \subseteq \R
    \}.
  \]
  If we define $\rho_k(f) = \|f \chi_{K}\|_1$,
  then $f \in L^1_\mathrm{loc}(\R)$
  if and only if
  $\rho_K(f) < \infty$. In fact,
  it is enough to consider
  $\rho_N(f) = \|f \chi_{[-N, N]}\|_1$.
  Given such a countable family
  of seminorms,
  one can always define a metric
  \[
    d(f, g)
    = \sum_{N=1}^\infty 2^{-N} \frac{\rho_N(f-g)}{1 + \rho_N(f-g)}
  \]
  which satisfies $d(f, f_k) \to 0$
  if and only if $\rho_N(f - f_k) \to 0$
  for every $N$.
\end{example}

\begin{remark}
  Suppose $\|\cdot\|$ is a norm
  on $L^1_\mathrm{loc}(\R)$, and
  consider $\chi_{[k, k + 1]}$.
  Then $\rho_N(\chi_{[k, k + 1]}) \to 0$, so
  \[
    c_k \chi_{[k, k + 1]} \longrightarrow 0
  \]
  in $L^1_\mathrm{loc}(\R)$
  for every choice of
  $c_k$. But then
  $\|c_k \chi_{[k, k + 1]}\| \to 0$,
  a contradiction for
  $c_k = 1 / \|\chi_{[k, k + 1]}\|$.
\end{remark}

\begin{example}
  For $C^\infty(\R)$, we can take
  $\rho_{N, n}(f) = \|f^{(n)}\chi_{[-N, N]}\|_\infty$.
  Then
  \[
    C^\infty(\R)^*
    = \{
      \text{continuous linear functionals }
      \mu : C^\infty(\R) \to \C
    \}
    = \mathcal{E}'(\R)
  \]
  is the space of compactly
  supported distributions, where
  $\mu$ is continuous if whenever
  $f_k \to f$ in $C^\infty(\R)$,
  we have $\mu(f_k) = \langle \mu, f_k \rangle \to \langle \mu, f \rangle$.
\end{example}

\begin{remark}
  For $\mathcal{D}'(\R) = C_c^\infty(\R)^*$,
  it is better to think of
  $C^\infty_c(\R) = \bigcup_{N=1}^\infty C_c^\infty([-N, N])$.
\end{remark}

\begin{definition}
  Let $\{\rho_\alpha\}_{\alpha \in J}$
  be a family of seminorms on a
  vector space $X$.
  \begin{enumerate}[(a)]
    \item $f_k \to f$ in $X$ means
      that $\rho_\alpha(f - f_k) \to 0$
      for every $\alpha \in J$.
    \item $X$ is \emph{Hausdorff}
      if $\rho_\alpha(f) = 0$ for
      every $f \in X$ if and only if
      $f = 0$.
  \end{enumerate}
\end{definition}

\begin{definition}
  Let $\{\rho_n\}_{n \in \N}$ be a
  countable, Hausdorff family
  of seminorms on $X$.
  If the metric
  \[
    d(f, g) = \sum_{n=1}^\infty 2^{-n} \frac{\rho_n(f-g)}{1 + \rho_n(f-g)}
  \]
  is complete, then we call
  $X$ a \emph{Fr\'echet space}.
\end{definition}

\begin{exercise}
  $\mathcal{S}(\R)$ and
  $C^\infty(\R)$ are Fr\'echet spaces
  (but $C_c^\infty(\R)$ is not).
\end{exercise}

\begin{remark}
  Recall that if
  $\mu : X \to \C$ is a linear
  functional on a normed space $X$,
  then $\mu$ is continuous if
  and only if $\mu$ is bounded
  (i.e. there exists $C > 0$ such that
  $|\langle f, \mu \rangle| \le C \|f\|$
  for every $f \in X$).
\end{remark}

\begin{theorem}
  Let $X$ be a Fr\'echet space
  with seminorms $\{\rho_n\}_{n \in \N}$
  and $\mu : X \to \C$ a linear
  functional. Then the following
  are equivalent:
  \begin{enumerate}
    \item $\mu$ is continuous, i.e.
      $f_k \to f$ implies
      $\langle f_k, \mu \rangle \to \langle f, \mu \rangle$
      as $k \to \infty$.
    \item $f_k \to 0$ in $X$
      implies $\langle f_k, \mu \rangle \to 0$.
    \item There exists $C > 0$ 
      and $N \in \N$ (depending
      on $\mu$) such that
      $|\langle f, \mu \rangle| \le C \sum_{n=1}^N \rho_n(f)$.
  \end{enumerate}
\end{theorem}

\begin{proof}
  $(1 \Leftrightarrow 2)$
  This follows from linearity.

  $(3 \Rightarrow 2)$ Suppose
  $f_k \to 0$, so
  $\rho_n(f_k) \to 0$ for every $n$. So
  \[
    |\langle f_k, \mu \rangle|
    \le C \sum_{n = 1}^N \rho_n(f_k)
    \xrightarrow[k \to \infty]{} 0,
  \]
  since
  each of the finitely many terms
  converge to $0$.

  $(2 \Rightarrow 3)$ We prove the
  contrapositive. Suppose (3) is
  false. So for every $C = N = k \in \N$,
  there exists $f_k \in X$ for which
  $|\langle f_k, \mu \rangle| > k \sum_{n = 1}^k \rho_n(f_k)$.
  Set $\varphi_k = f_k / |\langle f_k, \mu \rangle|$.
  Then
  \[
    \sum_{n = 1}^k \rho_n(\varphi_k)
    = \frac{1}{|\langle f_k, \mu \rangle|}
    \sum_{n = 1}^k \rho_n(f_k)
    < \frac{1}{|\langle f_k, \mu \rangle|} \cdot \frac{|\langle f_k, \mu \rangle|}{k}
    = \frac{1}{k}
    \xrightarrow[k \to \infty]{} 0.
  \]
  Thus $\rho_n(\varphi_k) \to 0$ as
  $k \to \infty$ for every $n \in \N$.
  So $\varphi_k \to 0$ in $X$.
  But $\langle \varphi_k, \mu \rangle = 1 \not\to 0$.
\end{proof}

\begin{example}
  What does the topology look like
  given a family of seminorms, i.e.
  what are the open sets?
  Consider seminorms
  $\rho_1(x_1, x_2) = |x_1|$ and
  $\rho_2(x_1, x_2) = |x_2|$. Consider
  strips
  \[
    B^i_r(x_1, x_2)
    = \{(y_1, y_2) :
    |y_i - x_i| < r\}
  \]
  for each seminorm $\rho_i$.
  Then the open balls are of the form
  \[
    B^1_r(x_1, x_2)
    \cap B^2_r(x_1, x_2).
  \]
  A basis in the general case
  involves intersections of
  finitely many strips (as in the
  product topology).
\end{example}

\section{Distributional Derivatives}

\begin{example}
  If $f, g$ are differentiable and
  decay at $\infty$ (think
  $f, g \in \mathcal{S}(\R)$), then
  \[
    \langle f, g' \rangle
    = \int_{-\infty}^\infty f(x) \overline{g'(x)} \, dx
    = \left.f(x) g(x)\right|_{x = -\infty}^\infty
    - \int_{-\infty}^\infty f'(x) \overline{g(x)} \, dx
    = -\langle f', g \rangle,
  \]
  by integration by parts, where
  $\left. f(x) g(x)\right|_{x = -\infty}^\infty = 0$ by the
    decay of $f, g$.
  Now define
  $\delta' : \mathcal{S}(\R) \to \C$ by
  \[
    \langle f, \delta' \rangle
    := - \langle f', \delta \rangle
    = - f'(0).
  \]
  for $f \in \mathcal{S}(\R)$. Then
  $\delta'$ is a linear functional,
  and
  \[
    |\langle f, \delta' \rangle|
    = |{-f'(0)}|
    \le \|f'\|_\infty
    = \|x^0 f^{(1)}(x)\|_\infty
    = \rho_{0, 1}(f)
    \le 1 \sum_{m = 0}^0 \sum_{n = 0}^1 \rho_{m, n}(f).
  \]
  Hence $\delta'$ is bounded on
  $\mathcal{S}(\R)$, so
  $\delta' \in \mathcal{S}'(\R) = \mathcal{S}(\R)^*$.
\end{example}

\begin{prop}
  Fix $\mu \in \mathcal{S}'(\R)$, and
  define $\mu' : \mathcal{S}(\R) \to \C$
  by
  \[
    \langle f, \mu' \rangle
    = -\langle f', \mu \rangle, \quad f \in \mathcal{S}(\R).
  \]
  Then $\mu' \in \mathcal{S}'(\R)$.
  In particular, every
  tempered distribution has a
  (distributional) derivative.
\end{prop}

\begin{proof}
  Since $\mu$ is bounded,
  there are $C, N > 0$ such that
  $|\langle f, \mu \rangle| \le C \sum_{m = 0}^M \sum_{n = 0}^N \rho_{m, n}(f)$.
  Then
  \[
    |\langle f, \mu' \rangle|
    = |\langle f', \mu \rangle|
    \le C \sum_{m = 0}^M \sum_{n = 0}^N \rho_{m, n}(f')
    = C \sum_{m = 0}^M \sum_{n = 0}^N
    \|x^m (f')^{(n)}\|_\infty \\
    = C \sum_{m = 0}^M \sum_{n = 0}^N
    \|x^m f^{n + 1}\|_\infty,
  \]
  so $\mu'$ is bounded. Thus $\mu' \in \mathcal{S}'(\R)$ also.
\end{proof}

\begin{example}
  Consider the Heaviside
  function $H = \chi_{[0, \infty)}$
  We can define
  \[
    \langle f, H \rangle
    = \int_{-\infty}^\infty f(x) \overline{H(x)} \, dx
    = \int_0^\infty f(x) \, dx.
  \]
  Then we can check that
  \begin{align*}
    |\langle f, H \rangle|
    &\le \int_0^\infty |f(x)|\, dx
    \le \int_0^1 |f(x)|\, dx
    + \int_1^\infty \frac{|f(x)|}{x^2} x^2 \, dx \\
    &\le \|f\|_\infty
    + \|x^2 f(x)\|_\infty \int_1^\infty \frac{1}{x^2} \, dx
    \le C \sum_{m = 0}^2 \sum_{n = 0}^0
    \|x^m f^{(n)}\|_\infty.
  \end{align*}
  Thus $H$ is bounded, so
  $H \in \mathcal{S}'(\R)$. 
  So $H$ has a distributional
  derivative $H' : \mathcal{S}(\R) \to \C$,
  where
  \[
    \langle f, H' \rangle
    := - \langle f', H \rangle
    = -\int_0^\infty f'(x)\, dx
    = -(f(\infty) - f(0))
    = f(0) = \langle f, \delta \rangle.
  \]
  In particular, we see that
  $H' = \delta$ as tempered
  distributions.
\end{example}

\begin{remark}
  Note that
  $L^1_\mathrm{loc}(\R) \subseteq \mathcal{D}'(\R) = C_c^\infty(\R)^*$
  and $L^1_\mathrm{loc}(\R) + \text{polynomial growth} \subseteq \mathcal{S}'(\R)$,
  so each of the functions
  in the latter
  has distributional derivatives.
\end{remark}

\begin{exercise}
  If $g$ is smooth, then
  $g'$ (as a tempered distribution)
  is the usual derivative.
\end{exercise}

\begin{remark}
  To avoid confusion, we sometimes
  write $D\mu$ for the
  distributional derivative
  and $g'$ for the pointwise a.e.
  derivative.
\end{remark}
