\chapter{Nov.~25 --- Distributions, Part 3}

\section{Space of Distributions}

\begin{remark}
  Let $\mathcal{D}'(\R) = C^\infty_c(\R)^*$.
  The topology on this
  space is \emph{not} given by a
  family of seminorms. Instead, we
  write $C^\infty_c(\R) = \bigcup_{K \text{ compact}} C^\infty(K)$.
  Note that $C^\infty(K)$ is a Fr\'echet space with seminorms
  \[
    \rho_{K, n}(f)
    = \|f^{(n)}(x) \chi_K(x)\|_\infty
  \]
  for $f \in C^\infty(K) = \{f \in C^\infty(\R) : \supp(f) \subseteq K\}$.
\end{remark}

\begin{theorem}
  $f_k \to f$ in $C^\infty_c(\R)$ if and
  only if there exists a compact set
  $K \subseteq \R$ such that
  $f_k \in C^\infty(K)$ and
  $f_k \to f$ in $C^\infty(K)$.
\end{theorem}

\begin{theorem}\label{thm:distribution-continuity}
  Given a linear function
  $\mu : C^\infty_c(\R) \to \C$,
  the following are equivalent:
  \begin{enumerate}
    \item $\mu$ is continuous;
    \item $\mu|_{C^\infty(K)}$ is
      continuous for each compact $K$;
    \item for all compact $K \subseteq \R$,
      there exists $C_K \ge 0$, $N_K \ge 0$
      such that
      \[
        |\langle f, \mu \rangle|
        \le C_K \sum_{n = 0}^{N_K}
        \|f^{(n)}(x) \chi_K(x)\|_\infty,
        \quad \text{for } f \in C^\infty(K).
      \]
  \end{enumerate}
\end{theorem}

\begin{definition}
  If there exists a single $N$ that
  can be used for each compact set $K$ in
  Theorem \ref{thm:distribution-continuity}(3),
  then we say that $\mu$ has
  \emph{finite order}. The \emph{order} of
  $\mu$ is the smallest such $N$.
\end{definition}

\begin{example}
  We have the following:
  \begin{enumerate}
    \item The order of $\delta^{(j)}$ is $j$.
    \item Consider the $\delta$-train
      $\mu = \sum_{n = - \infty}^\infty \delta_n$,
      so $\langle f, \mu \rangle = \sum_{n = -\infty}^\infty f(n)$.
      Note that this sum is finite since
      $f \in C^\infty(K)$ for some
      compact $K$. In fact,
      $|\langle f, \mu \rangle| \le C_K \|f\|_\infty$,
      so the order is $0$.
    \item Let
      $\nu = \sum_{n = 1}^\infty \delta_n^{(n)}$,
      so $\langle f, \nu \rangle = \sum_{n = 1}^\infty (-1)^n f^{(n)}(n)$.
      One can check that $\nu$ has
      infinite order.
  \end{enumerate}
\end{example}

\begin{remark}
  As sets, we have
  $C^\infty_c(\R) \subseteq \mathcal{S}(\R) \subseteq C^\infty(\R)$.
  What about their duals?
\end{remark}

\begin{theorem}
  We have the following:
  \begin{enumerate}
    \item $f_k \to f$ in $C^\infty_c(\R)$
      implies $f_k \to f$ in $\mathcal{S}(\R)$.
    \item $f_k \to f$ in $\mathcal{S}(\R)$
      implies $f_k \to f$ in $C^\infty(\R)$.
    \item $\mu \in \mathcal{S}'(\R)$
      implies $\mu|_{C^\infty_c(\R)} \in \mathcal{D}'(\R)$.
    \item $\mu \in \mathcal{E}'(\R)$
      implies $\mu|_{\mathcal{S}(\R)} \in \mathcal{S}'(\R)$.
  \end{enumerate}
  In particular, we have the
  containments
  $\mathcal{D}'(\R) \supseteq \mathcal{S}'(\R) \supseteq \mathcal{E}'(\R)$.
\end{theorem}

\section{Functions as Distributions}

\begin{theorem}
  $L^1_{\loc}(\R) \subseteq \mathcal{D}'(\R)$,
  where the embedding is given by
  \[
    \langle f, g \rangle
    = \int_{-\infty}^\infty f(x) \overline{g(x)} \, dx, \quad
    f \in C^\infty_c(\R).
  \]
\end{theorem}

\begin{proof}
  If $g \in L^1_{\loc}(\R)$
  and $f \in C^\infty(K)$, then
  \[
    \langle f, g \rangle
    \le \int_K |f(x) g(x)|\, dx
    \le \|f\|_\infty \|g \chi_K\|_{1}.
  \]
  Thus we see that $g$ defines a
  continuous linear functional on
  $C^\infty_c(\R)$.
\end{proof}

\begin{example}
  Note that $\delta \in \mathcal{D}'(\R)$,
  but it cannot be identified with a
  function in $L^1_{\loc}(\R)$.
\end{example}

\begin{example}
  Consider the space
  \begin{align*}
    M_b(\R)
    &= \{\text{bounded Radon measures on $\R$}\} \\
    &= \{\text{bounded locally finite Borel measures on $\R$}\}.
  \end{align*}
  If $\mu \in M_b(\R)$ and $f \in C^\infty(K)$,
  then we can define a linear functional by
  \[
    \langle f, \mu \rangle
    = \int f(x) \, d\mu(x).
  \]
  Then we have that
  \[
    |\langle f, \mu \rangle|
    \le \int_K |f(x)| \, d|\mu|(x)
    \le \|f\|_\infty |\mu|(K),
  \]
  which is finite as
  $\mu$ is a bounded Radon measure.
\end{example}

\begin{example}
  We have
  $1 / x \notin L^1_{\loc}(\R)$, but we
  can define
  $\pv(1 / x) : C^\infty_c(\R) \to \C$ by
  \[
    \langle f, \pv(1 / x) \rangle
    = \lim_{T \to \infty}\int_{1 / T \le |x| \le T} \frac{f(x)}{x}\, dx, \quad
    f \in C^\infty(K).
  \]
  One can show that the above limit
  exists, so $\pv(1 / x) \in \mathcal{D}'(\R)$.
\end{example}

\section{Operations on Distributions}

\begin{remark}
  Suppose $g \in L^1_{\loc}(\R)$, and set
  $T_a g(x) = g(x - a)$. If
  $f \in C^\infty_c(\R)$, then
  \[
    \langle f, T_a g \rangle
    = \int_{-\infty}^\infty f(x) \overline{g(x - a)} \, dx
    = \int_{-\infty}^\infty f(x + a) \overline{g(x)} \, dx
    = \langle T_{-a} f, g \rangle.
  \]
\end{remark}

\begin{definition}
  If $\mu \in \mathcal{D}'(\R)$, then
  define $T_a \mu : C^\infty_c(\R) \to \C$ by
  $\langle f, T_a \mu \rangle := \langle T_{-a} f, \mu \rangle$
  for $f \in C^\infty_c(\R)$.
\end{definition}

\begin{remark}
  Note that for $f \in C^\infty(K)$, we have
  \[
    |\langle f, T_a \mu \rangle|
    = |\langle T_{-a} f, \mu \rangle|
    \le C_{K + a} \sum_{n = 0}^{N_{K + a}} \|T_{-a} f\|_\infty
    \le C_{K + a} \sum_{n = 0}^{N_{K + a}} \|f\|_\infty
  \]
  since $T_{-a} f \in C^\infty(K + a)$.
  Thus $T_a \mu$ is a continuous
  linear functional on $C^\infty_c(\R)$,
\end{remark}

\begin{remark}
  One can similarly define other operations
  on distributions, e.g.
  dilation, modulation, etc.
  For example, the \emph{involution}
  $\widetilde{g}(x) = \overline{g(-x)}$
  satisfies
  \[
    \langle f, \widetilde{g} \rangle
    = \int f(x) \overline{\overline{g(-x)}} \, dx
    = \overline{\int \overline{f(x)} \overline{g(-x)} \, dx}
    = \overline{\int \overline{f(-x)} \overline{g(x)} \, dx}
    = \overline{\int \widetilde{f}(x) \overline{g(x)} \, dx}
    = \overline{\langle \widetilde{f}, g \rangle}.
  \]
  Thus if $\mu \in \mathcal{D}'(\R)$,
  then we can define
  $\langle f, \widetilde{\mu} \rangle := \overline{\langle \widetilde{f}, \mu \rangle}$
  for $f \in C^\infty_c(\R)$.
\end{remark}

\begin{example}
  Consider a translation of $\delta$:
  \[
    \langle f, T_a \delta \rangle
    := \langle T_{-a} f, \delta \rangle
    = T_{-a} f(0)
    = f(a).
  \]
  But sometimes by abuse of
  notation we may write
  \[
    \langle f, T_a\delta \rangle
    = \int f(x) \delta(x - a)\, dx
    = \int f(x + a) \delta(x)\, dx
    = f(a).
  \]
  This should be understood in the
  distributional sense, the
  above integral is \emph{not} a Lebesgue integral.
\end{example}

\section{Products and Convolution of Distributions}

\begin{remark}
  Suppose $g \in L^1_{\loc}(\R)$,
  $\theta \in C^\infty(\R)$. Then
  \[
    \langle f, \theta g \rangle
    = \int f(x) \overline{\theta(x) g(x)} \, dx
    = \int (f(x) \overline{\theta(x)}) \overline{g(x)} \, dx
    = \langle f \overline{\theta}, g \rangle,
    \quad f \in C^\infty_c(\R).
  \]
  Note that $f\overline{\theta} \in C^\infty_c(\R)$,
  so the above definition makes sense.
\end{remark}

\begin{definition}
  If $\mu \in \mathcal{D}'(\R)$
  and $\theta \in C^\infty(\R)$,
  then define $\theta \mu : C^\infty_c(\R) \to \C$ by
  $\langle f, \theta \mu \rangle
    := \langle f \overline{\theta}, \mu \rangle$.
\end{definition}

\begin{remark}
  We can check that for $f \in C^\infty(K)$
  (so $f \overline{\theta} \in C^\infty(K)$),
  \begin{align*}
    |\langle f, \theta \mu \rangle|
    = |\langle f \overline{\theta}, \mu \rangle|
    \le C_K \sum_{n = 0}^{N_K}
    \|(f \overline{\theta})^{(n)}\|_\infty
    &\le C_K \sum_{n = 0}^{N_K}
    \sum_{j = 0}^n \binom{n}{j} \|f^{(j)} \overline{\theta^{(n - j)}}\|_\infty \\
    &= C_K \sum_{n = 0}^{N_K} \sum_{j = 0}^n \binom{n}{j} \|\theta^{(n - j)}\|_\infty \|f^{(j)}\|_\infty.
  \end{align*}
  Thus we see that
  $\theta \mu \in \mathcal{D}'(\R)$.
\end{remark}

\begin{remark}
  Now consider $g \in L^1_{\loc}(\R)$,
  $f \in C^\infty_c(\R)$. Then we have
  \[
    (f * g)(x)
    = \int f(x - y) g(y) \, dy
    = \int \overline{\widetilde{f}(y - x)} g(y) \, dy
    = \overline{\langle T_x \widetilde{f}, g \rangle}.
  \]
\end{remark}

\begin{definition}
  The \emph{convolution} of
  $\mu \in \mathcal{D}'(\R)$
  with $f \in C^\infty_c(\R)$ is
  \[
    (f * \mu)(x)
    := \overline{\langle T_x \widetilde{f}, \mu \rangle}, \quad
    x \in \R.
  \]
\end{definition}

\begin{theorem}
  If $\mu \in \mathcal{D}'(\R)$
  and $f \in C^\infty_c(\R)$,
  then:
  \begin{enumerate}
    \item Convolution commutes with
      translation, i.e.
      $T_a(f * \mu) = (T_a f) * \mu = f * (T_a \mu)$.
    \item $f * \mu \in C^\infty(\R)$
      and $(f * \mu)' = f' * \mu$
      (and also $(f * \mu)' = f * D\mu$).
  \end{enumerate}
\end{theorem}

\begin{remark}
  Suppose $f, g, h$ are ``nice'' functions.
  Then
  \begin{align*}
    \langle f, g * h \rangle
    &= \int f(x) \overline{(g * h)(x)}\, dx
    = \iint f(x) \overline{g(x - y)} \overline{h(y)}\, dy dx \\
    &= \iint \left(f(x) \widetilde{g}(y - x) \, dx\right)
    \overline{h(y)}\, dy
    = \int (f * \widetilde{g})(y) \overline{h(y)}\, dy
    = \langle f * \widetilde{g}, h \rangle.
  \end{align*}
\end{remark}

\begin{theorem}
  If $\mu \in \mathcal{D}'(\R)$
  and $f, g \in C^\infty_c(\R)$, then
  $\langle f, g * \mu \rangle
    = \langle f * \widetilde{g}, \mu \rangle$.
\end{theorem}
