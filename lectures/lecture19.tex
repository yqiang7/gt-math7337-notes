\chapter{Nov.~6 --- Fourier Series}

\section{The Abstract Fourier Transform}

\begin{definition}
  A \emph{locally compact abelian (LCA) group}
  is an abelian group with a Hausdorff
  topology such that
  every point has a compact neighborhood,
  and $(x, y) \mapsto x + y$
  and $x \mapsto -x$ are continuous.
\end{definition}

\begin{example}
  Some examples of LCA groups are
  $\R^n$, $\Z^n$, $\T^n$, $\Z_N^n$,
  where $\T = [0, 1)$ with addition
  modulo $1$.
\end{example}

\begin{theorem}
  Every LCA group has a \emph{Haar measure},
  i.e. a nonzero Radon measure $\mu$
  which is translation-invariant.
\end{theorem}

\begin{definition}
  A \emph{character} on a LCA
  group is a continuous homomorphism
  \[\xi : G \to S^1 = \{z \in \C : |z| = 1\},\]
  i.e. a continuous map
  with $\xi(x + y) = \xi(x)\xi(y)$.
\end{definition}

\begin{definition}
  The \emph{dual group} of
  $G$ is $\widehat{G} = \{\xi : \xi \text{ is a character of $G$}\}$.
\end{definition}

\begin{example}
  Consider $\T = [0, 1)$. Fix $n \in \Z$,
  then $e_n(x) = e^{2\pi i n x}$ is a
  character on $\T$. Note that we have
  $e_n(x) = e_1(x)^n$, and one can show
  these are all of the characters
  on $\T$. Thus
  \[
    \widehat{\T}
    = \{e_n : n \in \Z\}
    \cong \Z.
  \]
\end{example}

\begin{example}
  The characters on $\Z$ are determined
  by their values at $1$. So
  for any $\xi \in [0, 1)$, we get a
  character $e_\xi(n) = e^{2\pi i \xi n}$.
  In particular, $\widehat{\Z} = \T$.
\end{example}

\begin{example}
  We have $\widehat{\Z}_N \cong \Z_N$
  and $\widehat{\R} \cong \R$.
  The characters on $\R$ are
  $e_\xi(x) = e^{2\pi i \xi x}$.
\end{example}

\begin{remark}
  The dual group $\widehat{G}$ is a
  LCA group under
  the multiplication of characters, where
  the topology on $\widehat{G}$
  is the topology of uniform convergence
  on compact sets (also called the compact-open topology).
\end{remark}

\begin{theorem}
  If $G$ is discrete, then $\widehat{G}$
  is compact. If $G$ is compact (so $G$ has
  finite measure),
  then $\widehat{G}$ is discrete,
  and the characters on $G$ are
  orthonormal in $L^2(G)$ (in fact, they
  form an orthonormal basis).
\end{theorem}

\begin{example}
  Consider the examples
  $\widehat{\Z} = \T$,
  $\widehat{\T} = \Z$,
  $\widehat{\R} = \R$,
  $\widehat{\Z}_N = \Z_N$.
\end{example}

\begin{remark}
  In some sense, $\R$ being non-compact
  and non-discrete makes things harder.
  On the other hand, having dilations in
  $\R$ can make harmonic analysis easier.
\end{remark}

\begin{theorem}[Pontryagin duality]
  ${\widehat{\widehat{G\,}}} \cong G$.
\end{theorem}

\begin{definition}
  The \emph{Fourier transform} of
  $f \in L^1(G)$ (complex-valued) is
  $\widehat{f} : \widehat{G} \to \C$ defined by
  \[
    \widehat{f}(\xi)
    = \int_G f(x) \overline{\xi(x)} \, d\mu(x), \quad
    \xi \in \widehat{G},
  \]
  where $d\mu$ is the Haar measure on
  $G$.
\end{definition}

\begin{example}
  On the real line, the Fourier transform
  is
  \[
    \widehat{f}(\xi)
    = \int_{\R} f(x) e^{-2\pi i \xi x} \, dx,
    \quad \xi \in \R.
  \]
  For $\T$, the Fourier coefficients are
  given by
  \[
    \widehat{f}(n)
    = \int_0^1 f(x) e^{-2\pi i n x} \, dx,
    \quad n \in \Z.
  \]
  For $\Z$, given $c = (c_n)_{n \in \Z}$,
  \[
    \widehat{c}(x)
    = \sum_{n = -\infty}^\infty c_n e^{-2\pi i n x}, \quad x \in \T
  \]
  For $\Z_N$, the discrete Fourier transform is
  \[
    \widehat{c}(k)
    = \sum_{n = 0}^{N - 1} c_n e^{-2\pi i n k / N},
    \quad k \in \Z_N.
  \]
\end{example}

\section{Fourier Series}

\begin{definition}
  The \emph{Fourier transform} of
  $f \in L^1(\T)$ is the sequence
  $\widehat{f} = (\widehat{f}(n))_{n \in \Z}$,
  where
  \[
    \widehat{f}(n) = \int_0^1 f(x) e^{-2\pi i n x} \, dx
  \]
  are the \emph{Fourier coefficients} of
  $f$. The \emph{inverse Fourier transform}
  of $f \in L^1(\T)$ is
  $\widecheck{f} = (\widecheck{f}(n))_{n \in \Z}$, where
  \[
    \widecheck{f}(n)
    = \int_0^1 f(x) e^{2\pi i n x} \, dx.
  \]
  Formally,
  $\sum_{n \in \Z} \widehat{f}(n) e^{2\pi i n x}$
  is the \emph{Fourier series} of $f$.
\end{definition}

\begin{remark}
  If $f \in L^2(\T)$ and we accept that
  $\{e_n\}_{n \in \Z}$ is an orthonormal
  basis for $L^2(\T)$, then
  \[
    f(x) = \sum_{n = -\infty}^\infty \widehat{f}(n) e^{2\pi i n x}
    = \sum_{n = -\infty}^\infty \langle f, e_n \rangle e_n,
  \]
  where the convergence is in $L^2(\T)$
  (we do not get pointwise convergence in
  general).
\end{remark}

\begin{remark}
  If $f \in L^1(\T)$, then we have
  \[
    |\widehat{f}(n)|
    \le \int_0^1 |f(x) e^{-2\pi i n x}| \, dx
    = \|f\|_{L^1},
  \]
  so we see that
  $\widehat{f} \in \ell^\infty(\Z)$.
\end{remark}

\begin{remark}
  If we define
  $\widecheck{c}(x) = \sum_{n = -\infty}^\infty c_n e^{2\pi i n x}$,
  then we can view the Fourier series as
  \[
    (\widehat{f})^{\vee}(x)
    = \sum_{n = -\infty}^\infty \widehat{f}(n) e^{2\pi i n x}.
  \]
\end{remark}

\begin{example}
  For $0 < \alpha \le 1$, consider
  the function
  \[
    g(x) = \sum_{m = 0}^\infty 2^{-\alpha m} e^{2\pi i 2^m x}, \quad x \in \T
  \]
  which converges absolutely in
  $C(\T)$.
  One can show that $g$ is
  nowhere differentiable.

  Weierstrass first showed that
  $g(x) = \sum_{m = 0}^\infty a^{-m} \cos(b^m x)$
  is nowhere differentiable for
  $a, b$ large enough.
\end{example}

\section{Partial Sums of Fourier Series}
\begin{definition}
  For $f \in L^1(\T)$, define the
  \emph{partial sums}
  \[
    S_N f(x)
    = \sum_{n = -N}^N \widehat{f}(n) e^{2\pi i n x}.
  \]
\end{definition}

\begin{remark}
  Consider a discrete characteristic function
  \[
    \chi_N(n) =
    \begin{cases}
      1 & |n| \le N,\\
      0 & |n| > N.
    \end{cases}
  \]
  Note that $\chi_N \in \ell^1(\Z)$.
  Then we can compute that
  \[
    \widehat{\chi}_N(x)
    = \widecheck{\chi}_N(x)
    = \sum_{n = -\infty}^\infty \chi_N(n) e^{2\pi i n x}
    = \sum_{n = -N}^N e^{2\pi i n x}
    = \frac{\sin((2N + 1) \pi x)}{\sin(\pi x)} = d_N(x)
  \]
  by the geometric series formula. Note
  that $d_N \in C(\T) \subseteq L^1(\T)$.\footnote{Note that $\T$ is compact, so a continuous function on $\T$ is automatically integrable.}
  We also have
  $\|d_N\|_1 \to \infty$.
\end{remark}

\begin{remark}
  Using the \emph{Dirichlet kernel}
  $d_N$, we can write the partial sums as
  \begin{align*}
    S_N f(x)
    &= \sum_{n = -N}^N \widehat{f}(n) e^{2\pi i n x}
    = \sum_{n = -N}^N
    \left(\int_0^1 f(t) e^{-2\pi i n t} \, dt\right) e^{2\pi i n x} \\
    &= \int_{0}^1 f(t) \sum_{n = -N}^N e^{2\pi i n (x - t)} \, dt
    = \int_0^1 f(t) d_N(x - t) \, dt
    = (f * d_N)(x).
  \end{align*}
  Intuitively,
  $S_N f = (\widehat{f} \chi_N)^\vee = (\widehat{f})^{\vee} * \widecheck{\chi}_N = f * d_N$,
  but the above is a direct proof.
\end{remark}

\section{Approximate Identities on \texorpdfstring{$\T$}{T}}

\begin{definition}
  An \emph{approximate identity} on
  $\T$ is a sequence
  $\{k_n\}_{n \in \N}$ such that
  \begin{enumerate}
    \item $k_n \in L^1(\T)$ and
      $\displaystyle \int_{0}^1 k_n(x)\, dx = 1$;
    \item $\sup_{n \in \N} \|k_n\|_1 < \infty$;
    \item for any $\delta > 0$,
      $\displaystyle \int_{|x| \ge \delta} |k_n(x)| \, dx \to 0$.
  \end{enumerate}
\end{definition}

\begin{remark}
  The Dirichlet kernel $d_N$ does not
  form an approximate identity.
\end{remark}

\begin{definition}
  For $f \in L^1(\T)$, define the
  \emph{Ces\`aro sums}
  \[
    \sigma_N f(x)
    = \frac{S_0 f(x) + S_1 f(x) + \cdots + S_N f(x)}{N + 1}.
  \]
  Note that this is the average of the
  partial sums.
\end{definition}

\begin{remark}
  We can rewrite the Ces\`aro sums as
  \[
    \sigma_N f(x)
    = \sum_{n = -N}^N \underbrace{\left(1 - \frac{|n|}{N + 1}\right)}_{W_N(n)} \widehat{f}(n) e^{2\pi i n x}
    = (f * \widehat{W}_N)(x).
  \]
  We define the \emph{Fej\'er kernel}
  to be
  \[
    w_N(x)
    = \widehat{W}_N(x)
    = \frac{1}{N + 1}
    \left(\frac{\sin((N + 1) \pi x)}{\sin(\pi x)}\right)^2.
  \]
\end{remark}
