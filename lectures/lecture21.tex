\chapter{Nov.~13 --- Fourier Series, Part 3}

\section{Shannon Sampling Theorem}

\begin{definition}
  Define the \emph{Paley-Wiener space}
  to be
  \[
    \mathrm{PW}(\R)
    = \mathcal{F} L_{[-1 / 2, 1 / 2]}^2(\R)
    = \{f \in L^2(\R) : \supp(\widehat{f}) \subseteq [-1 / 2, 1 / 2]\}.
  \]
\end{definition}

\begin{remark}
  Set $e_n(x) = e^{-2\pi i nx} \chi_{[-1 / 2, 1 / 2]}(x) = M_n \chi_{[-1 / 2, 1 / 2]}(x)$,
  then
  \[
    \widecheck{e}_n(x)
    = T_n \widecheck{\chi}_{[-1 / 2, 1 / 2]}(x)
    = \frac{\sin(\pi(x - n))}{\pi(x - n)}.
  \]
  Since $\{e_n\}$ is an orthonormal
  basis for $L^2_{[-1 / 2, 1 / 2]}(\R)$,
  the $\{\widecheck{e}_n\}$ form an orthonormal basis for $\mathrm{PW}(\R)$.
  Note that if $f \in \mathrm{PW}(\R)$,
  then we can write
  \[
    f(x)
    = \sum_{n = -\infty}^\infty
    \langle f, \widecheck{e}_n \rangle
    \widecheck{e}_n(x).
  \]
  By the unitarity of the Fourier transform,
  we can compute that
  \[
    \langle f, \widecheck{e}_n \rangle
    = \langle \widehat{f}, e_n \rangle
    = \int_{-1 / 2}^{1 / 2} \widehat{f}(\xi) e^{2\pi i n \xi} \, d\xi
    = \int_{-\infty}^{\infty} \widehat{f}(\xi) e^{2\pi i n \xi} \, d\xi
    = (\widehat{f})^\vee(n)
    = f(n).
  \]
  Note that the third equality follows
  from $\supp(\widehat{f}) \subseteq [-1 / 2, 1 / 2]$. Thus
  \[
    f(x) = \sum_{n = -\infty}^\infty
    f(n) \frac{\sin(\pi(x - n))}{\pi(x - n)}.
  \]
  This is called the
  \emph{Shannon sampling theorem}
  (or \emph{classical sampling theorem}),
  i.e. that any function in $\mathrm{PW}(\R)$
  is completely determined by its
  values on the integers.
\end{remark}

\begin{remark}
  Fix $b > 0$, and set
  $e_{nb}(x) = e^{2\pi i nbx}$.
  This is an orthogonal basis for
  $L^2[0, 1 / b]$.
\end{remark}

\begin{example}
  If we set $b = 1 / 2$, then we can write
  \[
    \{e_{n / 2}\}_{n \in \Z}
    = \{e_n\}_{n \in \Z}
    \cup \{e_{(n + 1) / 2}\}_{n \in \Z}.
  \]
  This is a union of two orthonormal
  bases for $L^2[0, 1]$.
  For $f \in L^2[0, 1]$, we can write
  \[
    2 f
    = \sum_{n = -\infty}^\infty
    \langle f, e_n \rangle e_n
    + \sum_{n = -\infty}^\infty
    \langle f, e_{(n + 1) / 2} \rangle
    e_{(n + 1) / 2}
    = \sum_{n \in \Z}
    \langle f, e_{n / 2} \rangle
    e_{n / 2}.
  \]
\end{example}

\begin{remark}
  If we take $0 < b < 1$ and
  consider $e^{2\pi i n bx} = e_{bn}(x)$
  in $L^2[0, 1]$, then for
  $f \in L^2[0, 1] \subseteq L^2[0, 1 / b]$,
  \[
    f = \frac{1}{b}
    \sum_{n = -\infty}^\infty
    \langle f, e_{bn} \rangle
    e_{bn}
  \]
  in $L^2[0, 1 / b]$. Since
  $f = 0$ on $[1, 1 / b]$, we have
  the same expansion in
  $L^2[0, 1]$. Note, however,
  that this expansion is not orthogonal
  and not unique. These sets
  $\{e_{bn}\}_{n \in \Z}$ are called
  \emph{frames}.
\end{remark}

\section{Weyl's Equidistribution Theorem}

\begin{theorem}\label{thm:weyl-equidistribution}
  If $\alpha \in \T$ is irrational, then
  $\{k \alpha\}_{k \in \N}$
  is \emph{equidistributed} in
  $[0, 1)$, i.e.
  \[
    \lim_{N \to \infty}
    \frac{\#\{1 \le k \le N : k\alpha \Mod{1} \in (a, b)\}}{N}
    = \lim_{N \to \infty} \frac{1}{N} \sum_{k = 1}^N \chi_{(a, b)}(k \alpha)
    = b - a
  \]
  for every $0 \le a < b \le 1$.
\end{theorem}

\begin{remark}
  Consider $f \in C(\T)$. Then we can
  view
  $\frac{1}{N}\sum_{k = 1}^N f(k \alpha)$
  as some type of Riemann sum, so we might
  expect that
  \[
    \frac{1}{N} \sum_{n = 1}^N f(k\alpha)
    \xrightarrow[N \to \infty]{} \int_0^1 f(x) \, dx.
  \]
  The \emph{Birkhoff ergodic theorem} says
  that the above average in fact converges
  to the integral (given that
  $\{k\alpha\}$ is equidistributed).
  In fact, the result holds for
  any $f \in L^1(\T)$.
\end{remark}

\begin{exercise}
  Prove the above statement for
  $e_n(x) = e^{2\pi i n x}$
  (and thus for trigonometric polynomials
  $p(x) = \sum_{n = -N}^N c_n e^{2\pi i nx}$).
  Then apply density to prove this
  for $f \in C(\T)$.
\end{exercise}

\begin{proof}[Proof of Theorem \ref{thm:weyl-equidistribution}]
  Approximate $\chi_{(a, b)}$ by
  a trapezoidal continuous
  function $f$ which equals
  $\chi_{(a, b)}$ on
  $(a + \epsilon, b - \epsilon)$
  and decays linearly to $0$ on
  $[a, a + \epsilon]$ and
  $[b - \epsilon, b]$. Then
  \[
    b - a - \epsilon
    = \int_0^1 f(x)\, dx
    = \lim_{N \to \infty} \frac{1}{N} \sum_{k = 1}^N f(k \alpha)
    \le \liminf_{N \to \infty}
    \frac{1}{N} \sum_{k = 1}^N \chi_{(a, b)}(k \alpha).
  \]
  Taking a similar approximation with
  $g$ which equals
  $\chi_{(a, b)}$ on
  $[a, b]$ and decays linearly to $0$ on
  $[a  - \epsilon, a]$ and
  $[b, b + \epsilon]$, we also get an
  upper bound:
  \[
    \limsup_{N \to \infty}
    \frac{1}{N} \sum_{k = 1}^N \chi_{(a, b)}(k \alpha)
    = \lim_{N \to \infty}
    \frac{1}{N} \sum_{k = 1}^N g(k \alpha)
    \le \int_0^1 g(x) \, dx
    = b - a + \epsilon.
  \]
  This holds for arbitrary $\epsilon$,
  hence the limit is equal to $b - a$.
\end{proof}

\begin{corollary}[Kronecker's theorem]
  $\{k \alpha \Mod{1}\}_{k \in \N}$
  is dense in $[0, 1)$.
\end{corollary}

\section{\texorpdfstring{$L^p$}{Lp}-Convergence of Fourier Series}

\begin{remark}
  Recall that $\{e_n\}_{n \in \Z}$ is
  complete in $L^p(\T)$ and $C(\T)$, i.e.
  $\mathrm{span}\{e_n\}_{n \in \Z}$ is
  dense.

  \pagebreak
  For $1 \le p < \infty$, if
  $f \in L^p(\T)$, do the partial sums
  \[
    S_N f(x)
    = \sum_{n = -N}^N \widehat{f}(n) e^{2\pi i n x}
    = (f * d_N)(x)
  \]
  converge to $f$ in $L^p$-norm?

  The motivation is as follows.
  Suppose that $S_N f \to f$
  in $L^p$ for every $f \in L^p(\T)$.
  Then we have a linear operator
  $S_N : L^p(\T) \to L^p(\T)$ on a
  Banach space. Moreover, the convergence
  is also pointwise. Thus for any individual
  $f \in L^p(\T)$,
  \[
    \sup_{N \in \N} \|S_N f\|_{p}
    < \infty.
  \]
  Then by the Banach-Steinhaus theorem (also
  called the uniform boundedness principle),
  we get
  \[
    \sup_{N \in \N} \|S_N\|
    = \sup_{N \in \N} \sup_{\|f\|_p = 1}
    \|S_N f\|_p
    < \infty
  \]
  In fact, this is equivalent to the
  $L^p$-convergence of Fourier series,
  and idea behind the proof is to instead
  argue about the uniform boundedness
  of $S_N$.

  Note that $\|S_N f\|_p = \|f * d_N\|_p \le \|f\|_p \|d_N\|_1$,
  so $\|S_N\| \le \|d_N\|_1 \to \infty$.

  For $p = 1$, we can argue as follows.
  Fix an approximate identity
  $\{k_\lambda\}$ with
  $k_\lambda \ge 0$, so that
  \[
    \|k_\lambda\|_1
    = \int_0^1 |k_\lambda|
    = \int_0^1 k_\lambda
    = 1.
  \]
  Then we can see that
  \[
    \|S_N\|
    \ge \|S_N k_\lambda\|_1
    = \|k_\lambda * d_N\|_1
    \xrightarrow[\lambda \to \infty]{} \|d_N\|_1
  \]
  since $k_\lambda * d_N \to d_N$ in $L^1$.
  Thus $\|S_N\| \ge \|d_N\|_1 \to \infty$,
  so the partial sums cannot converge
  in $L^1$. In particular,
  $\{e_n\}_{n \in \N}$ is \emph{not}
  a Schauder basis for $L^1(\T)$.
  A similar argument works to show that
  $\{e_n\}_{n \in \N}$ is not a Schauder
  basis for $C(\T)$. However,
  one has the following:
\end{remark}

\begin{theorem}
  $\{e_n\}_{n \in \Z}$ is a
  Schauder basis for $L^p(\T)$
  for $1 < p < \infty$, with the
  ordering
  \[
    \Z = \{0, -1, 1, -2, 2, \ldots\}.
  \]
\end{theorem}

\begin{example}[Haar system]
  Another example of a Schauder basis
  is the following. Let
  \[
    \varphi = \chi_{[0, 1)} \quad \text{and} \quad
    \psi = \chi_{[0, 1 / 2)} - \chi_{[1 / 2, 1)}.
  \]
  Note that $\langle \varphi, \psi \rangle = 0$.
  Then we can define
  \[
    \psi_1 = 2^{1 / 2} \psi(2x)
    \quad \text{and} \quad
    \psi_2 = 2^{1 / 2} \psi(2x - 1),
  \]
  which are still orthogonal with
  $\varphi, \psi$ and with each other.
  Continuing this process, we get an
  orthonormal basis for $L^2(\T)$.
  Moreover, this is a Schauder basis
  for $L^p(\T)$ for $1 < p < \infty$
  and it is unconditional, i.e.
  the convergence does not depending
  on the ordering of the functions.
\end{example}

\begin{remark}
  There also exists a continuous analogue
  of the Haar system, the functions
  are called $D_4$ and $W_4$.
  The functions $\varphi, \psi$ from
  the Haar system are known as
  $D_2, W_2$. These are called
  \emph{wavelets}.
\end{remark}
