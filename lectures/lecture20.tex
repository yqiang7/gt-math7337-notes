\chapter{Nov.~11 --- Fourier Series, Part 2}

\section{Approximate Identities on \texorpdfstring{$\T$}{T}, Continued}
\begin{remark}
  Using a similar proof as in the
  case for $\R$, for $f \in L^1(\T)$, we have
  \[
    \widehat{f}
    \in c_0(\Z) = \{c = (c_n) : c_n \to 0 \text{ as } |n| \to \infty\}.
  \]
  However, in general
  $\widehat{f} \notin \ell^1(\Z)$.
\end{remark}

\begin{exercise}
  Let $d_N$ be the Dirichlet kernel on
  $\T$. Show that
  \[
    \frac{4}{\pi^2}
    \sum_{k = 1}^N \frac{1}{k}
    \le \|d_N\|_1
    \le 3 + \frac{4}{\pi^2} \sum_{k = 1}^N \frac{1}{k}.
  \]
  In particular, $\|d_N\|_1 \to \infty$ as $N \to \infty$,
  so $d_N$ does not form an approximate
  identity.
\end{exercise}

\begin{remark}
  Define the function
  \[
    W_N(x)
    =
    \begin{cases}
      1 - |n| / (N + 1) & |n| \le N,\\
      0 & |n| > N.
    \end{cases}
  \]
  Note that $W_N = (\chi_N * \chi_N) / (2N + 1)$
  and $\widehat{W}_{2N} = (\widehat{\chi}_N)^2 = d_N^2 / (2N + 1)$.
  The Fej\'er kernel is
  \[
    w_N = \widehat{W}_N
    = \frac{1}{N} \left(\frac{\sin((N + 1)\pi x)}{\sin(\pi x)}\right)^2,
  \]
  and one can check that
  $\int_0^1 w_N = 1$ and
  $\|w_N\|_1 = 1$.
\end{remark}

\begin{theorem}
  $(f * \widecheck{W}_N)(x) = (f * w_N)(x)$.
\end{theorem}

\begin{exercise}
  $\{w_N\}_{N \in \N}$ forms an
  approximate identity on $\T$.
\end{exercise}

\begin{remark}
  Unlike the real line, we cannot just
  take any $L^1$ function and dilate it
  to form an approximate identity on $\T$,
  as the dilations need not be $1$-periodic.
\end{remark}

\begin{exercise}
  If $\{k_N\}$ is an approximate identity
  on $\T$, then:
  \begin{enumerate}
    \item For $1 \le p < \infty$,
      $f \in L^p(\T)$ implies
      $f * k_N \to f$ in $L^p$-norm.
    \item $f \in C(\T)$ implies
      $f * k_N \to f$ uniformly.
  \end{enumerate}
\end{exercise}

\section{The Inversion Formula on \texorpdfstring{$\T$}{T}}

\begin{theorem}
  If $f \in L^1(\T)$ and $\widehat{f} \in \ell^1(\Z)$, then
  \[
    f(x) = (\widehat{f})^\vee(x)
    = \sum_{n = -\infty}^\infty \widehat{f}(n) e^{2\pi i n x},
  \]
  where the above series converges
  uniformly in $C(\T)$.
\end{theorem}

\begin{proof}
  Note that $(\widehat{f})^\vee(x) = \sum_{n = -\infty}^\infty \widehat{f}(n) e^{2\pi i n x}$
  converges absolutely in
  the norm of $C(\T)$ since
  \[
    \sum_{n = -\infty}^\infty \|\widehat{f}(n) e^{2\pi i n x}\|_{\infty}
    = \sum |\widehat{f}(n)|
    < \infty.
  \]
  In particular, this shows that
  $(\widehat{f})^\vee \in C(\T)$.
  Also $f * w_N \to f$ in $L^1$-form, where
  \[
    (f * w_N)(x)
    = \sum_{n = -N}^N \left(1 - \frac{|n|}{N + 1}\right) \widehat{f}(n) e^{2\pi i n x}.
  \]
  Note that $(1 - |n|/(N + 1)) \widehat{f}(n) \to \widehat{f}(n)$
  pointwise as $N \to \infty$, and
  \[
    \left|\left(1 - \frac{|n|}{N + 1}\right)\widehat{f}(n)\right|
    \le |\widehat{f}(n)|
    \in \ell^1(\Z),
  \]
  so by the dominated convergence theorem
  (for counting measure)
  \[
    (f * w_N)(x) \xrightarrow[N \to \infty]{} \sum_{n = -\infty}^\infty \widehat{f}(n) e^{2\pi i n x} = (\widehat{f})^\vee(x)
  \]
  pointwise.
  Since $f * w_N \to f$ in $L^1$-norm,
  there exists a subsequence
  $f * w_{N_k} \to f$ pointwise a.e.,
  hence we see that
  $f = (\widehat{f})^\vee$ a.e.
  Since $f$ and $(\widehat{f})^\vee$ are
  continuous, we get that
  $f = (\widehat{f})^\vee$ everywhere.
\end{proof}

\begin{corollary}
  If $f \in L^1(\T)$ and
  $\widehat{f}(n) = 0$ for every $n$,
  then $f = 0$ a.e. Moreover, if
  $f, g \in L^1(\T)$ and
  $\widehat{f}(n) = \widehat{g}(n)$
  for all $n$, then
  $f = g$ a.e.
\end{corollary}

\begin{definition}
  The \emph{Fourier algebra}
  (or \emph{Wiener algebra}) on $\T$ is
  \[
    A(\T)
    = \{f \in L^1(\T) : \widehat{f} \in \ell^1(\Z)\}.
  \]
\end{definition}

\begin{remark}
  The Fourier algebra
  $A(\T)$ is closed under
  convolution and forms a Banach algebra.
  Moreover, it is a dense subspace of
  $C(\T)$.
\end{remark}

\section{\texorpdfstring{$L^2$}{L2}-Convergence of Fourier Series}

\begin{remark}
  Let $e_n(x) = e^{2\pi i n x}$. The
  functions $e_n$ are orthonormal in
  $L^2(\T)$: For $n \ne m$,
  \[
    \langle e_m, e_n \rangle
    = \int_0^1 e^{2\pi i m x} \overline{e^{2\pi i n x}} \, dx
    = \int_0^1 e^{2\pi i (m - n) x} \, dx
    = \left.\frac{e^{2\pi i (m - n) x}}{2\pi i (m - n)}\right|_{x = 0}^1
      = 0.
  \]
\end{remark}

\begin{theorem}
  Let $\{f_n\}_{n \in \N}$ be an
  orthonormal sequence in a Hilbert
  space $H$. Then the following
  are equivalent:
  \begin{enumerate}
    \item $\{f_n\}_{n \in \N}$ is
      \emph{complete} (i.e. $\cspan\{f_n\} = H$).
    \item $f = \sum_{n = 1}^\infty c_n f_n$
      in $H$
      for a unique choice of scalars $c_n$.
      That is, $\{f_n\}$ is a
      \emph{Schauder basis} for $H$.
    \item $f = \sum_{n = 1}^\infty \langle f, f_n \rangle f_n$,
      where the convergence is in $H$.
    \item (Plancherel)
      $\|f\|^2 = \sum_{n = 1}^\infty |\langle f, f_n \rangle|^2$.
    \item (Parseval)
      $\langle f, g \rangle = \sum_{n = 1}^\infty \langle f, f_n \rangle \langle f_n, g \rangle$.
  \end{enumerate}
\end{theorem}

\begin{example}
  The above theorem does not hold when
  the sequence is not orthonormal.
  Recall the \emph{Weierstrass approximation theorem}:
  $\{x^n\}_{n = 0}^\infty$ is complete
  in $C(\T)$. But not every
  $f \in C(\T)$ can be written
  as $f(x) = \sum_{n = 0}^\infty c_n x^n$
  (such functions are infinitely
  differentiable in some disk).
\end{example}

\begin{theorem}
  We have the following:
  \begin{enumerate}
    \item For $1 \le p < \infty$,
      $\{e_n\}_{n \in \N}$ is
      complete in $L^p(\T)$.
    \item $\{e_n\}_{n \in \N}$ is
      complete in $C(\T)$.
  \end{enumerate}
\end{theorem}

\begin{proof}
  (1) If $f \in L^p(\T)$, then we have
  \[
    (f * w_N)
    = \sum_{n = -N}^N \left(1 - \frac{|n|}{N + 1}\right) \widehat{f}(n) e^{2\pi i n x}
    \xrightarrow[N \to \infty]{} f
  \]
  in $L^p$-norm. But
  $f * w_n \in \mathrm{span}\{e_n\}_{n \in \Z}$,
  so $\cspan\{e_n\}_{n \in \Z} = L^p(\T)$.
  The same proof works for (2).
\end{proof}

\begin{corollary}
  $\{e_n\}_{n \in \Z}$ is an orthonormal
  basis for $L^2(\T)$. Furthermore,
  \[
    f = \sum_{n = -\infty}^\infty \widehat{f}(n) e^{2\pi i n x},
  \]
  where the above convergence is
  in $L^2$-norm. Moreover, one has
  \begin{enumerate}
    \item (Plancherel) $\|f\|_2^2
      = \sum_{n = -\infty}^\infty |\widehat{f}(n)|^2$,
    \item (Parseval)
      $\langle f, g \rangle = \langle \widehat{f}, \widehat{g} \rangle
      = \sum_{n = -\infty}^\infty \widehat{f}(n) \overline{\widehat{g}(n)}$,
  \end{enumerate}
  so the Fourier transform
  operator
  $\mathcal{F} : L^2(\T) \to \ell^2(\Z)$
  given by $f \mapsto \widehat{f}$
  is unitary.
\end{corollary}

\begin{theorem}
  If $1 < p < \infty$, then
  $S_N f = (f * d_N)(x) = \sum_{n = -N}^N \widehat{f}(n) e^{2\pi i n x}$
  converges in $L^p$-norm, so
  \[
    f = \sum_{n = -\infty}^\infty \widehat{f}(n) e^{2\pi i n x}
  \]
  in $L^p$-norm with respect to the
  ordering $\Z = \{0, -1, 1, -2, 2, -3, 3, \dots\}$.
  However, the convergence is conditional:
  Only this ordering (and finite
  permutations of it) need converge
  when $p \ne 2$.
\end{theorem}
