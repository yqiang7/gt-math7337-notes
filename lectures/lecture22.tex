\chapter{Nov.~18 --- Fourier Series, Part 4}

\section{The Poisson Summation Formula}

\begin{remark}
  Suppose $f \in L^1(\R)$
  Set $\varphi(x) = \sum_{n = -\infty}^\infty f(x + n)$.
  Then by Fubini-Tonelli,
  \[
    \|\varphi\|_1
    = \int_0^1 \left|\sum_{n = -\infty}^\infty f(x + n)\right|
    \le 
    \sum_{n = -\infty}^\infty \int_0^1 |f(x + n)|
    = \sum_{n = -\infty} \int_{n}^{n + 1} |f|
    = \int_{-\infty}^\infty |f|
    = \|f\|_1 < \infty,
  \]
  so $\varphi \in L^1(\T)$.
  Moreover, the integrals are the
  same if we do not take absolute
  values.
\end{remark}

\begin{exercise}
  Let $f, \varphi$ as above.
  Prove that
  \[
    \widehat{\varphi}(n)
    = \int_0^1 \sum_{n = -\infty}^\infty f(x) e^{-2\pi i n x} \, dx
    = \widehat{f}(n).
  \]
\end{exercise}

\begin{theorem}[Poisson summation]
  Let $f : \R \to \C$.
  If there exist $C, \epsilon > 0$
  such that
  \[
    |f(x)| \le \frac{C}{(1 + |x|)^{1 + \epsilon}}
    \quad \text{and} \quad
    |\widehat{f}(x)| \le \frac{C}{(1 + |x|)^{1 + \epsilon}},
  \]
  then for every $x \in \R$,
  \[
    \sum_{n = -\infty}^\infty f(x + n)
    = \sum_{n = -\infty}^\infty \widehat{f}(n) e^{2\pi i n x}.
  \]
  In particular, setting $x = 0$
  one obtains
  \[
    \sum_{n = -\infty}^\infty f(n)
    = \sum_{n = -\infty}^\infty \widehat{f}(n).
  \]
\end{theorem}

\begin{proof}
  Both $f, \widehat{f} \in L^1(\R)$,
  hence $f, \widehat{f} \in C_0(\R)$
  by Fourier inversion.
  Also $\varphi \in L^1(\T)$ and
  \[
    \sum_{n = -\infty}^\infty
    |\widehat{\varphi}(n)|
    = \sum_{n = -\infty}^\infty
    |\widehat{f}(n)|
    < \infty,
  \]
  so $\widehat{\varphi} \in \ell^1(\Z)$.
  Therefore the inversion formula gives
  \[
    \varphi(x) = \sum_{n = -\infty}^\infty \widehat{\varphi}(n) e^{2\pi i n x}
  \]
  pointwise. (In fact one can
  get uniform convergence on compact
  sets.)
\end{proof}

\begin{remark}
  Formally consider
  $\mu = \sum_{n = -\infty}^\infty \delta_n$,
  where $\delta_n$ is the
  Dirac delta at $n$. Then
  \[
    \langle f, \mu \rangle
    = \int_{-\infty}^\infty f(x)\, d\mu(x)
    = \sum_{n = -\infty}^\infty f(n).
  \]
  A similar computation shows that
  \[
    \langle \widehat{f}, \mu \rangle
    = \sum_{n = -\infty}^\infty \widehat{f}(n).
  \]
  These series are equal by
  Poisson summation, so
  if the Parseval identity holds,
  then one gets
  \[
    \langle f, \widecheck{\mu} \rangle
    = \langle \widehat{f}, \mu \rangle
    = \sum_{n = -\infty}^\infty \widehat{f}(n)
    = \sum_{n = -\infty}^\infty f(n)
    = \langle f, \mu \rangle.
  \]
  Thus Poisson summation says that
  $\mu = \widecheck{\mu}$.
\end{remark}

\section{Wiener's Lemma}

\begin{remark}
  Let
  $C(\T) = \{f \text{ continuous on } \T\}$,
  which is closed under products and
  \[
    \|fg\|_\infty \le \|f\|_\infty \|g\|_\infty.
  \]
  In particular,
  $C(\T)$ is a commutative
  Banach algebra. Also note
  that if $f \in C(\T)$ and
  $f(x) \ne 0$ for every $x$, then
  $1 / f \in C(\T)$. We say that
  $C(\T)$ is \emph{inverse-closed}.

  Recall that the Wiener (Fourier)
  algebra is
  \[
    A(\T) = \{f \in L^1(\T) : \widehat{f} \in \ell^1(\Z)\}
    = \{\widehat{c} : c = (c_n) \in \ell^1(\Z)\}.
  \]
  The equality holds by the
  formulas $f(x) = \sum_{n = -\infty}^\infty \widehat{f}(n) e^{2\pi i n x}$
  and $\widecheck{c}(x) = \sum_{n = -\infty}^\infty c_n e^{2\pi i n x}$.
  Note that
  $A(\T) \subseteq C(\T)$ is
  dense in the uniform norm.

  On the other hand, if we define
  \[
    \|f\|_{A(\T)}
    = \sum_{n = -\infty}^\infty |\widehat{f}(n)|,
  \]
  then for any $f, g \in A(\T)$,
  we have
  \[
    f g = (\widehat{fg})^\vee
    = (\widehat{f} * \widehat{g})^\vee.
  \]
  Since $\widehat{f}, \widehat{g} \in \ell^1(\Z)$,
  we have $\widehat{f} * \widehat{g} \in \ell^1(\Z)$,
  so $fg \in A(\T)$. Moreover,
  \[
    \|fg\|_{A(\T)}
    = \|\widehat{f} * \widehat{g}\|_1
    \le \|\widehat{f}\|_1 \|\widehat{g}\|_1
    = \|f\|_{A(\T)} \|g\|_{A(\T)}.
  \]
  Thus we see that
  $A(\T)$ is also a commutative
  Banach algebra
  with respect to pointwise products.

  If $A(\T)$ inverse-closed?
  In other words, if $f \in A(\T)$
  and $f(x) \ne 0$ for all $x$,
  must $1 / f \in A(\T)$?
\end{remark}

\begin{lemma}\label{lem:poly-AT}
  If $P(\xi) = \sum_{k = -N}^N a_k e^{2\pi i k \xi}$,
  then
  $\|P\|_{A(\T)} \le (2N + 1)^{1 / 2} \|P\|_\infty$.
\end{lemma}

\begin{proof}
  By Cauchy-Schwarz, we have
  \[
    \|P\|_{A(\T)}
    = \sum_{k = -N}^N |a_k| \cdot 1
    \le (2N + 1)^{1 / 2}
    \left(\sum_{k = -N}^N |a_k|^2\right)^{1 / 2}
    = (2N + 1)^{1 / 2} \|P\|_2
  \]
  where the second equality
  is by Plancherel's theorem.
  The result follows since
  $\|P\|_2 \le \|P\|_\infty$ on
  $\T$.
\end{proof}

\pagebreak
\begin{lemma}[Wiener's lemma]
  If $g \in \ell^1(\Z)$ and
  $\widehat{g}(\xi) \ne 0$ for any
  $\xi$, then there is
  $h \in \ell^1(\Z)$ such that
  \[
    \widehat{h}(\xi)
    = \frac{1}{\widehat{g}(\xi)},
    \quad \xi \in \T.
  \]
  Equivalently, if
  $G \in A(\T)$ and
  $G(\xi) \ne 0$ for any $\xi$, then
  $1 / G \in A(\T)$.
\end{lemma}

\begin{proof}
  Assume $g \in \ell^1(\Z)$ and
  $\widehat{g}(\xi) \ne 0$ for any
  $\xi$. Let $G = \widehat{g}$.
  We prove the result in two steps:
  \begin{enumerate}
    \item Assume $0 \le G \le 1$.
      Then since $G$ is nonzero,
      \[
        d = \inf_{\xi \in \T} G(\xi) > 0.
      \]
      Let $H = 1 - G = \widehat{\delta} - \widehat{g}$,
      where $\delta$ is the
      delta sequence. Note that
      \[
        \widehat{\delta}(\xi)
        = \sum_{n = -\infty}^\infty \delta_n e^{2\pi i n \xi}
        = \delta_0 e^{2\pi i 0 \xi}
        = 1.
      \]
      Then $H = \widehat{h} - \widehat{g} = (\delta - g)^\wedge \in A(\T)$.
      So if we let
      $h = \delta - g \in \ell^1(\Z)$,
      then
      $\widehat{h} = H$, and
      \[
        \|H\|_\infty
        = \|1 - G\|_\infty
        = 1 - d < 1.
      \]
      Hence
      $\sum_{n = 0}^\infty H(\xi)^n$
      converges, and
      $\sum_{n = 0}^\infty H^n = 1 / (1 - H(\xi)) = 1 / G(\xi) \in C(\T)$.

      However, we need convergence
      in $A(\T)$. Fix
      $0 < \epsilon < d / 2$.
      Let $p = h \chi_{[-N, N]}$ for
      $N$ large enough so that
      $\|p - h\|_1 < \epsilon$.
      Let $P = \widehat{p} = \sum_{n = -N}^N h(n) e^{2\pi i n \xi}$.
      Let $r = p - h$
      and $R = P - H$. Then
      \begin{align*}
        \|P\|_\infty
        &= \|H + R\|_\infty
        \le \|H\|_\infty + \|R\|_\infty
        \le (1 - d) + \|P - H\|_\infty \\
        &\le (1 - d) + \|\widehat{p} - \widehat{h}\|_\infty
        \le (1 - d) + \|p - h\|_1
        = 1 - d + \epsilon.
      \end{align*}
      Now we can compute that
      \[
        \|H^n\|_{A(\T)}
        = \|(P - R)^n\|_{A(\T)}
        = \left\|
        \sum_{j = 0}^n \binom{n}{j} P^{j} (-R)^{n - j}
        \right\|_{A(\T)}
      \]
      where we can use the
      binomial theorem since
      $A(\T)$ is a commutative
      Banach algebra. Then
      \[
        \|H^n\|_{A(\T)}
        \le \sum_{j = 0}^n \binom{n}{j}
        \|P\|_{A(\T)}^{j} \|R\|_{A(\T)}^{n - j}
        \le \sum_{j = 0}^n \binom{n}{j}
        (2Nj + 1)^{1 / 2}
        \|P^j\|_\infty \epsilon^{n - j},
      \]
      where the second equality is
      by Lemma \ref{lem:poly-AT}
      and $\|R\|_{A(\T)} = \|p - h\|_1 < \epsilon$.
      Thus
      \[
        \|H^n\|_{A(\T)}
        \le (2Nn + 1)^{1 / 2}
        \sum_{j = 1}^n \binom{n}{j}
        \|P\|_\infty^j \epsilon^{n - j}
        = (2Nn + 1)^{1 / 2}
        (\|P\|_\infty + \epsilon)^n,
      \]
      where $\|P^j\|_\infty \le \|P\|_\infty^j$
      by submultiplicativity. Since
      $\|P\|_\infty \le 1 - d + \epsilon$,
      \[
        \|H^n\|_{A(\T)}
        \le (2Nn + 1)^{1 / 2}
        (1 - d + 2\epsilon)^n,
      \]
      where $1 - d + 2\epsilon < 1$.
      So
      $\sum_{n = 0}^\infty \|H^n\|_{A(\T)} < \infty$,
      so $\sum_{n = 0}^\infty H^n$
      converges in $A(\T)$ to
      $1 / (1 - H) = 1 / G$.
  \end{enumerate}
  One can reduce the general case to
  this first case, see the course
  notes.
\end{proof}

\section{Distributions}

\begin{remark}
  Let $1 \le p < \infty$.
  Note that
  $L^p(\R)^* \cong L^{p'}(\R)$,
  where $L^p(\R)^*$ is the dual
  space of $L^p(\R)$, i.e. the
  space of bounded linear functions
  on $L^p(\R)$. If we fix
  $g \in L^{p'}(\R)$, then we can define
  \[
    \mu_g(f)
    = \langle f, g \rangle
    = \int_{-\infty}^\infty f(x) \overline{g(x)} \, dx,
  \]
  which is linear in $f$ and
  antilinear in $g$ (this is known
  as a \emph{sesquilinear form}).

  Recall that
  $X^* = \{\text{bounded linear functionals $\mu$ on $X$}\}$,
  which is equivalent to continuous
  linear functionals when
  $X$ is a Banach space.
  We consider the following
  space of distributions:
\end{remark}

\begin{definition}
  Define the following:
  \begin{itemize}
    \item $\mathcal{D}'(\R) = C_c^\infty(\R)^*$,
      the space of
      \emph{distributions}.
    \item $\mathcal{S}'(\R) = \mathcal{S}(\R)^*$,
      the space of
      \emph{tempered distributions}.
    \item $\mathcal{E}'(\R) = C^\infty(\R)^*$,
      the space of
      \emph{compactly supported distributions}.
  \end{itemize}
\end{definition}

\begin{example}
  If $f \in C_c^\infty(\R)$,
  then define
  $\langle f, \delta \rangle = f(0)$.
  This is the Dirac delta as
  a distribution.
\end{example}

\begin{remark}
  Recall that the Schwartz space is
  \[
    \mathcal{S}(\R)
    = \{f \in C^\infty(\R) : x^m f^{(n)}(x) \in L^\infty(\R)\}.
  \]
  This is \emph{not} a Banach space
  (it does not have a norm).
  But we can define seminorms
  \[
    \rho_{m, n}(f)
    = \|x^m f^{(n)}\|_\infty, \quad
    m, n \ge 0.
  \]
  There is no way to combine these
  seminorms into a single norm.\footnote{Compare this with $C^1_b(\R) = \{f : \|f\|_\infty, \|f'\|_\infty < \infty\}$, where one can define a norm $\|f\|_{C^1_b} = \|f\|_\infty + \|f'\|_\infty$.}
  However, one can define a metric:
  \[
    d(f, g)
    = \sum_{m = 0}^\infty \sum_{n = 0}^\infty
    2^{-m - n}
    \frac{\rho_{m, n}(f - g)}{1 + \rho_{m, n}(f - g)}, \quad f, g \in \mathcal{S}(\R).
  \]
  Note that convergence with respect
  to $d$ is the same as
  convergence in $\rho_{m, n}$ for
  every $m, n$:
  \[
    \lim_{k \to \infty}
    \rho_{m, n}(f - f_k) = 0.
  \]
\end{remark}

\begin{definition}
  We say that $f_k \to f$ in
  $\mathcal{S}(\R)$ if for every
  $m, n \ge 0$,
  \[
    \rho_{m, n}(f - f_k) = \|x^m f^{(n)}(x) - x^m (f_k)^{(n)}(x)\|_\infty \longrightarrow 0
  \]
  Therefore, a linear functional
  $\mu : \mathcal{S}(\R) \to \C$
  is \emph{continuous} if
  $f_k \to f$ in $\mathcal{S}(\R)$
  implies $\langle f_k, \mu \rangle \to \langle f, \mu \rangle$.
\end{definition}

\begin{remark}
  Since convergence in $\mathcal{S}(\R)$
  is a very strong condition,
  we expect that it is easy for
  $\mu$ to satisfy the above
  condition. This intuitively explains
  why $\mathcal{S}'(\R)$ is so large.
\end{remark}

\begin{example}
  Assume that
  $f_k \to f$ in $\mathcal{S}(\R)$.
  Then $\langle f_k, \delta \rangle = f_k(0)$, so
  \[
    |\langle f - f_k, \delta \rangle|
    = |f(0) - f_k(0)|
    \le \rho_{0, 0}(f - f_k)
    \longrightarrow 0.
  \]
  Thus we see that $\delta \in \mathcal{S}'(\R) = \mathcal{S}(\R)^*$.
\end{example}
