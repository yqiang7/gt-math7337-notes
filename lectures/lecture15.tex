\chapter{Oct.~23 --- The Paley-Wiener Theorem}

\section{Local Uncertainty Principle}

\begin{remark}
  Recall the classical uncertainty principle:
   For $f \in L^2(\R)$,
      $\|f\|_2^2 \le 4\pi \|xf(x)\|_2^2 \|\xi \widehat{f}(\xi)\|_2^2$.
    There is also a local version of
    this statement:
\end{remark}

\begin{theorem}[Local uncertainty principle]
  If $f \in L^2(\R)$ and $\epsilon > 0$,
  then for any $\xi_0 \in \R$, we have
  \[
    \int_{\xi_0 - \epsilon}^{\xi_0 + \epsilon} |\widehat{f}(\xi)|^2\, d\xi
    \le 8 \pi \epsilon \|f\|_2 \|x f(x)\|_2.
  \]
\end{theorem}

\section{The Paley-Wiener Theorem}

\begin{remark}
  Recall the smoothness and decay
  theorems: If $x, x f(x) \in L^1(\R)$, then
  $(\widehat{f})'$ exists. ``Extreme decay''
  corresponds to compact support, where
  $\supp(f) \subseteq [-T, T]$ if
  $f = 0$ a.e. outside $[-T, T]$.
  When this is the case, the
  \emph{Paley-Wiener theorem}
  says that $\widehat{f}$ has
  extreme smoothness:
  $\widehat{f}$ extends to an analytic
  function on $\C$.
\end{remark}

\begin{remark}
  There is also motivation from
  Fourier series. The Fourier transform
  on $\R$ is
  \[
    \widehat{f}(\xi)
    = \int_{-\infty}^\infty f(x) e^{-2\pi i \xi x}\, dx.
  \]
  The functions $e_\xi(x) = e^{2\pi i \xi x} : \R \to S^1 \subseteq \C$
  are continuous group homomorphism, and
  $dx$ is a Haar measure on $\R$
  (invariant under the group action).
  For $\Z$, Haar measure is the counting
  measure, and for
  $c = (c_n)_{n \in \Z} \in \ell^1$,
  we can define its Fourier transform
  to be
  \[
    \widehat{c}(\xi)
    = \sum_{n = -\infty}^\infty c_n e^{-2\pi i n \xi}, \quad \xi \in \R.
  \]
  In fact, we can take $\xi \in [0, 1)$.
  Also note that $\widehat{c}$ is $1$-periodic.

  Suppose now that $c$ is compactly
  supported, say only $c_1, \dots, c_N$ are
  nonzero. Then
  \[
    \widehat{c}(\xi)
    = \sum_{n = 1}^N c_n e^{-2\pi i n \xi}
    = \sum_{n = 1}^N c_n (e^{-2\pi i \xi})^n
  \]
  If we set $z = e^{-2\pi i \xi} \in S^1$,
  then the above becomes
  \[
    \widehat{c}(z)
    = \sum_{n = 1}^N c_n z^n, \quad |z| = 1,
  \]
  which is a polynomial. In particular,
  $\widehat{c}$ extends to an
  analytic function for all $z \in \C$.
\end{remark}

\begin{remark}
  Suppose $f \in L^2(\R)$ and
  $\supp(f) \subseteq [-T, T]$
  (we say that $f$ is ``time-limited''
  to $[-T, T]$). Then
  \[
    \int_{-\infty}^\infty |f(x)|\, dx
    = \int_{-T}^T |f(x)|\, dx
    \le \left(\int_{-T}^T |f(x)|^2\, dx\right)^{1/2}
      \left(\int_{-T}^T 1^2\, dx\right)^{1/2}
    = (2T)^{1 / 2} \|f\|_2 < \infty,
  \]
  so $f \in L^1(\R)$. Thus we can take its
  Fourier transform as usual:
  \[
    \widehat{f}(\xi)
    = \int_{-T}^T f(x) e^{-2\pi i \xi x}\, dx,
    \quad \xi = \alpha + i\beta \in \C.
  \]
  Note that $e^{-2\pi i \xi x} = e^{-2\pi i \alpha x} e^{2\pi \beta x}$,
  where $e^{2\pi \beta x} > 0$. Then
  \[
    \int_{-T}^T |f(x) e^{-2\pi i \xi x}|\, dx
    = \int_{-T}^T |f(x)| e^{2\pi \beta x}\, dx
    \le e^{2\pi |\beta| T} \int_{-T}^T |f(x)|\, dx
    \le (2T)^{1 / 2} e^{2\pi |\beta| T} \|f\|_2.
  \]
  So $\widehat{f}$ can be extended to
  all of $\C$. We will see that this
  extension is analytic.

  We now have a function
  $\widehat{f} : \C \to \C$ satisfying
  \[
    |\widehat{f}(\xi)|
    \le (2T)^{1 / 2} e^{2\pi |\beta| T} \|f\|_2,
    \quad \xi = \alpha + i\beta \in \C.
  \]
  Note that $f \in L^1(\R)$ implies
  $\widehat{f} \in C_0(\R)$, so $\widehat{f}$
  is bounded on $\R$. However, the
  extension to $\C$ cannot be bounded
  (for non-constant $f$) by Liouville's
  theorem.
\end{remark}

\begin{definition}
  A function $F : \C \to \C$ has
  \emph{exponential type $T$} if
  there exists $A > 0$ such that
  \[
    |F(\xi)|
    \le A e^{T|\xi|},
    \quad \xi \in \C.
  \]
\end{definition}

\begin{remark}
  So from the above discussion,
  $\widehat{f}$ has exponential type
  $2\pi T$.
  We now show that $\widehat{f}$ is
  analytic on $\C$. Recall that $F$
  is analytic if and only if $F'$ exists
  on $\C$. We can compute
  \[
    (\widehat{f})'(\xi)
    = \lim_{\substack{\eta \to 0 \\ \xi \in \C}}
    \frac{\widehat{f}(\xi + \eta) - \widehat{f}(\xi)}{\eta}
    = \lim_{\eta \to 0}
    \int_{-T}^T f(x) \frac{e^{-2\pi i (\xi + \eta) x} - e^{-2\pi i \xi x}}{\eta}\, dx.
  \]
  We have seen that $f$ is integrable,
  and the exponential term is bounded
  on $[-T, T]$ by the mean value theorem.
  So by the dominated convergence theorem,
  we can exchange the limit and integral
  to get
  \begin{align*}
    (\widehat{f})'(\xi)
    &= \int_{-T}^T f(x) \lim_{\eta \to 0}
    \frac{e^{-2\pi i (\xi + \eta) x} - e^{-2\pi i \xi x}}{\eta}\, dx \\
    &= \int_{-T}^T (-2\pi i x) f(x) e^{-2\pi i \xi x}\, dx
    = ((-2\pi i x) f(x))^\wedge(\xi).
  \end{align*}
  Note that $xf(x) \in L^1(\R)$ and also has
  support in $[-T, T]$, so the above
  exists for any $\xi \in \C$.
\end{remark}

\begin{theorem}[Paley-Wiener theorem]
  We have the following:
  \begin{enumerate}
    \item If $f \in L^2(\R)$ and
      $\supp(f) \subseteq [-T, T]$,
      then $\widehat{f}$ extends to an
      analytic function on $\C$
      with exponential type $2\pi T$.
    \item If $F : \C \to \C$ is analytic
      with exponential type $2\pi T$,
      then there exists $f \in L^2(\R)$
      with $\supp(f) \subseteq [-T, T]$
      and $\widehat{f}(\xi) = F(\xi)$
      for $\xi \in \R$, i.e.
      $\widehat{f} = F|_{\R}$.
  \end{enumerate}
\end{theorem}

\begin{proof}
  (1) This was the above discussion.

  (2) The proof is more difficult and
  uses a lot of complex analysis.
  See Katznelson's book.
\end{proof}

\begin{remark}
  Recall that analytic functions have
  Taylor expansions. Suppose
  $F : \C \to \C$ and $F = 0$ on a line
  segment. Then if $\eta$ lies on the
  line segment, we can write
  \[
    F(\xi)
    = \sum_{k = 0}^\infty F^{(k)}(\eta) \frac{(\xi - \eta)^k}{k!}.
  \]
  Since $F^{(k)}(\eta) = 0$ for all $k$,
  we get
  $F \equiv 0$ on all of $\C$. In fact,
  if $F = 0$ on any set with an accumulation
  point, then $F \equiv 0$ on $\C$.
\end{remark}

\begin{corollary}
  If $f \in L^2(\R)$, $f \ne 0$, has
  compact support,
  then $\widehat{f}(\xi) = 0$ for
  only countably many $\xi$.
\end{corollary}

\begin{proof}
  If $\widehat{f}(\xi) = 0$ for uncountably
  many $\xi$, then there exists
  $\xi_n \to \xi$ such that
  $\widehat{f}(\xi_n) = 0$ for all $n$
  (split $\C$ into countably many compact
  sets and apply a countability argument).
  Hence $\widehat{f}, f = 0$.
\end{proof}

\begin{corollary}
  If $f \in L^2(\R)$, then
  $f$ and $\widehat{f}$ are both compactly
  supported if and only if $f = 0$.
\end{corollary}

\begin{proof}
  Suppose $f$ is compactly supported,
  then $\widehat{f}$ is analytic on $\C$.
  Suppose $\widehat{f}(\xi) = 0$ a.e. on
  $\R$
  outside $[-\Omega, \Omega]$. Since
  $\widehat{f}$ is analytic and
  $\R \setminus [-\Omega, \Omega]$
  has an accumulation point, we get
  $\widehat{f}, f = 0$.
\end{proof}

\begin{remark}
  In particular, the above
  corollary implies that if
  $f \in C^\infty_c(\R)$ is nonzero, then
  $\widehat{f}$ is not compactly supported.
  But $C_c^\infty(\R) \subseteq \mathcal{S}(\R)$
  and $\mathcal{F}$ maps
  $\mathcal{S}(\R)$ onto $\mathcal{S}(\R)$,
  so we do know $\mathcal{F}(C_c^\infty(\R)) \subseteq \mathcal{S}(\R)$.
\end{remark}

\section{Hilbert-Schmidt Operators}

\begin{remark}
Let $H$ be a separable, infinite-dimensional
Hilbert space. Recall the following
definitions and results from functional
analysis:
\end{remark}

\begin{definition}
  A linear operator $A : H \to H$ is
  \emph{compact} if $A(D)$ is contained
  in a compact set, where
  $D = \{f \in H : \|f\| \le 1\}$
  is the closed unit disk in $H$.
\end{definition}

\begin{theorem}[Spectral theorem for compact, self-adjoint operators]
  If $A : H \to H$ is a compact, self-adjoint
  operator, then there exists
  an orthonormal basis
  $\{e_n\}_{n \in \N}$ of $H$ such that
  \[
    f = \sum_{n = 1}^\infty \langle f, e_n \rangle e_n,
    \quad
    A f = \sum_{n = 1}^\infty \lambda_n \langle f, e_n \rangle e_n,
    \quad
    A e_n = \lambda_n e_n.
  \]
  Moreover, $\lambda_n \to 0$ if there
  are infinitely many $\lambda_n \ne 0$.
\end{theorem}

\begin{remark}
  Note that $\|A e_n\| = \|\lambda_n e_n\| = |\lambda_n| \to 0$,
  so the identity operator $I$ is not compact.
\end{remark}

\begin{example}
  An example of a compact box in
  $\R^\N$ looks like
  \[
    [0, 1] \times [0, 1 / 2]
    \times [0, 1 / 4] \times \cdots.
  \]
\end{example}

\begin{definition}
  A linear operator $A : H \to H$ is
  \emph{Hilbert-Schmidt} if there exists
  an orthonormal basis
  $\{e_n\}_{n \in \N}$ such that
  $\sum_{n = 1}^\infty \|A e_n\|^2 < \infty$.
\end{definition}

\begin{remark}
  For an operator $A : H \to H$, if we
  write
  \[
    A f(x) =
    \int_{-\infty}^\infty k(x, y) f(y)\, dy,
  \]
  then $A$ is Hilbert-Schmidt if
  and only if $k \in L^2(\R^2)$.
\end{remark}
