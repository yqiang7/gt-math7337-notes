\chapter{Sept.~25 --- Kernels and Schwartz Space}

\section{More Kernels}

\begin{remark}
  Recall that if $\{k_\lambda\}$ (where
  $k_\lambda(x) = \lambda k(\lambda x)$)
  is an
  approximate identity, then
  $f * k_\lambda \to f$ in $L^p$-norm.
  We also have
  \[
    (f * k_\lambda)^\wedge
    = \widehat{f} \cdot \widehat{k}_\lambda.
  \]
  Note that $\widehat{k}_\lambda(\xi) = \widehat{k}(\xi/\lambda)$,
  so $\widehat{k}_\lambda(\xi) \to 1$
  pointwise.
  If the Dirichlet kernel $d_{2\pi \lambda}(\xi) = \sin(2\pi \lambda \xi) / (\pi \xi)$
  were integrable, then we would have
  $\widehat{d}_{2\pi \lambda}(\xi) = \chi_{[-\lambda, \lambda]}(\xi)$,
  so we must have
  $d_{2\pi \lambda} \notin L^1(\R)$.

  Some alternatives are the following:
  \begin{enumerate}
    \item 
      The
      Fej\'er kernel $w_\lambda(\xi) = d^2_{2\pi \lambda}(\xi) \in L^1(\R)$.
      We saw this in the proof of the
      inversion formula.
    \item The \emph{de la Vall\'ee Poussin kernel}
      $v_{\lambda}$, which has
      $\widehat{v}_\lambda$ as a trapezoid
      which is $1$ on $[-\lambda, \lambda]$
      and decays linearly to $0$
      on $[-2\lambda, -\lambda]$ and $[\lambda, 2\lambda]$.
      One has
      \[
        (f * v_\lambda)^\wedge
        = \widehat{f} \cdot \widehat{v}_\lambda
        = \widehat{f}
        \quad \text{on } [-\lambda, \lambda].
      \]
      Explicitly, one can define
      $v(x) = 2w_2(x) - w(x)$ and
      $v_\lambda(x) = \lambda v(\lambda x)$.
    \item The \emph{Poisson kernel}
      $p(x) = 1 / \pi(x^2 + 1)$.
    \item The \emph{Gauss kernel}
      $\phi(x) = e^{-\pi x^2}$.
  \end{enumerate}
\end{remark}

\begin{exercise}
  Show that if
  $f \in L^1(\R)$, then
  $\supp(\widehat{f})$ is compact
  if and only if $f = f * g$ for some
  $g \in L^1(\R)$.
  Hint: Use the de la Vall\'ee Poussin kernel.
\end{exercise}

\begin{exercise}
  Let $\Phi = \widehat{\phi}$, where
  $\phi$ is the Gauss kernel. Show that
  $\Phi'(\xi) = -2\pi \xi \Phi(\xi)$.
  Then solve this differential equation
  to get $\Phi(\xi) = \Phi(0) \phi(\xi)$.
  Finally show that $\Phi(0) = 1$, and
  conclude $\Phi(\xi) = \phi(\xi)$.
\end{exercise}

\begin{remark}
  There are other ways to find functions
  which are their own Fourier transforms.
  The inversion formula says that
  if $f, \widehat{f} \in L^1(\R)$,
  then we have
  \[
    f(x)
    = (\widehat{f})^\vee(x)
    = f^{\wedge \wedge}(-x)
    = f^{\wedge \wedge \wedge \wedge}(x).
  \]
  In particular, for $f$ sufficiently
  nice, if we take
  $g = f + f^{\wedge} + f^{\wedge \wedge} + f^{\wedge \wedge \wedge}$,
  then $\widehat{g} = g$.
\end{remark}

\begin{theorem}[Weierstrass approximation theorem]
  If $f \in C[a, b]$ and $\epsilon > 0$,
  then there exists a polynomial $p$
  such that $\|f - p\|_{\infty} = \sup_{x \in [a, b]} |f(x) - p(x)| < \epsilon$.
\end{theorem}

\begin{proof}
  Fix $f \in C[a, b]$, and choose
  $[a, b] \subseteq (-R, R)$.
  Extend $f$ to a function $g \in C_0(\R)$ on
  $\R$ which equals $f$ on $[a, b]$
  and is supported in $(-R, R)$.
  Let $\phi$ be the Gauss kernel, and
  choose $\lambda$ so that
  \[
    \|g - g * \phi_\lambda\|_\infty < \frac{\epsilon}{2}.
  \]
  Note that $\phi_\lambda(x) = \lambda e^{-\pi \lambda^2 x^2}$
  is analytic
  and has a Taylor expansion
  \[
    \phi_{\lambda}(x)
    = \sum_{n = 0}^\infty \lambda \frac{(-\pi \lambda^2 x^2)^n}{n!}
    = \sum_{n = 0}^\infty \frac{(-1)^n \pi^n \lambda^{2n + 1}}{n!} x^{2n}.
  \]
  Set $q(x) = \sum_{n = 0}^N \frac{(-1)^n \pi^n \lambda^{2n + 1}}{n!} x^{2n}$.
  If $N$ is large enough, then we will have
  \[
    \sup_{|x| \le 2R} |\phi_\lambda(x) - q(x)|
    < \frac{\epsilon}{2 \|g\|_1}
  \]
  since the Taylor series converges
  uniformly on compact sets. Then
  \[
    |(g * \phi_\lambda)(x) - (g * q)(x)|
    \le \int_{-R}^R |g(y)| |\phi_{\lambda}(x - y) - q(x - y)| \, dy
    \le \frac{\epsilon}{2\|g\|_1} \int_{-R}^R |g(y)| \, dy
    < \frac{\epsilon}{2}
  \]
  for $|x| \le R$. So
  $|f(x) - (g * q)(x)| \le \epsilon$ for
  $x \in [a, b]$. Finally, observe that
  \[
    (g * q)(x)
    = \sum_{n = 0}^N \frac{(-1)^n \pi^n \lambda^{2n + 1}}{n!} \int g(y) (x - y)^{2n}\, dy,
  \]
  which is a polynomial by the binomial
  theorem. So we can take $p = g * q$.
\end{proof}

\section{Schwartz Space}

\begin{definition}
  Define the \emph{Schwartz space}
  $\mathcal{S}(\R)$ to be
  \[
    \mathcal{S}(\R)
    = \{
      f \in C^\infty(\R) :
      x^m f^{(n)}(x) \in L^\infty(\R)
      \text{ for all } m, n \ge 0
    \}.
  \]
\end{definition}

\begin{remark}
  Note that $|x^m f^{(n)}(x)| \le C_{m, n}$
  for some constant $C_{m, n}$, so
  \[
    |f^{(n)}(x)| \le \frac{C_{m, n}}{|x|^m}.
  \]
  In particular, for every $n$ and polynomial $p$,
  there exists a constant $C_{n, p}$ such that
  \[
    |f^{(n)}(x)| \le \frac{C_{n, p}}{|p(x)|}.
  \]
  Note, however, that this does
  not imply $f$ has exponential decay.
\end{remark}

\begin{remark}
  Note that $\rho_{m, n}(f) = \|x^m f^{(n)}\|_\infty$
  defines a \emph{seminorm} for each $m, n \ge 0$, but not a norm.\footnote{Recall that a \emph{seminorm} $\rho$ is a function satisfying $0 \le \rho(f) < \infty$, $\rho(cf) = |c| \rho(f)$, and $\rho(f + g) \le \rho(f) + \rho(g)$. If we additionally have $\rho(f) = 0$ if and only if $\rho = 0$, then $\rho$ is a \emph{norm}.}
\end{remark}

\begin{remark}
  Let $f \in \mathcal{S}(\R)$. Then
  \begin{align*}
    \|x^m f^{(n)}\|_1
    &= \int_{|x| \le 1} |x^m f^{(n)}(x)| \, dx
    + \int_{|x| \ge 1} |x^m f^{(n)}(x)| \, dx \\
    &\le 2 \|x^m f^{(n)}\|_\infty
    + \int_{|x| \ge 1} \frac{|x^{m + 2} f^{(n)}(x)|}{|x|^2} \, dx
    \le 2 \|x^m f^{(n)}\|_\infty
    + C \|x^{m + 2} f^{(n)}\|_\infty.
  \end{align*}
  In particular, if the
  $L^\infty$-norms are controlled, then
  so are the $L^1$-norms.
\end{remark}

\begin{exercise}
  Recall the smoothness and decay theorems:
  The Fourier transform interchanges
  smoothness and decay. Write
  $Df = f'$, and show that
  \[
    (D^n((-2\pi i x)^m f(x)))^\wedge(\xi)
    = (2\pi i \xi)^n D^m \widehat{f}(\xi).
  \]
  Note that we have
  \[
    D^n((-2\pi i x)^m f(x))
    = \sum_{j = 0}^n \binom{n}{j} D^j (-2\pi i x)^m f^{(n - j)}(x),
  \]
  so in particular, the Schwartz condition
  on $f$
  implies the Schwartz condition on
  $\widehat{f}$.
\end{exercise}

\begin{theorem}
  If $f \in \mathcal{S}(\R)$, then
  $\widehat{f} \in \mathcal{S}(\R)$.
  Moreover, we have
  $f = (\widehat{f})^\vee$ by the inversion
  formula, so the
  Fourier transform $\mathcal{F} : \mathcal{S}(\R) \to \mathcal{S}(\R)$
  by $f \mapsto \widehat{f}$
  is a bijection.
\end{theorem}

\begin{remark}
  Note that $C^\infty_c(\R) \subsetneq \mathcal{S}(\R)$, and
  we will see later that
  $\mathcal{F}(C^\infty_c(\R)) \subseteq \mathcal{S}(\R) \setminus C^\infty_c(\R)$.
\end{remark}
