\chapter{Nov.~4 --- Energy Concentration, Part 2}

\section{Energy Concentration, Continued}

\begin{remark}
  Recall the following problem.
  Fix $T, \Omega > 0$. We want to determine
  \[
    E_{T, \Omega}
    = \sup \left\{
      \int_{-T}^T |f|^2 :
      \|f\|_2 = 1, \supp(\widehat{f}) \subseteq [-\Omega, \Omega]
    \right\}.
  \]
  We previously defined
  $A_T f = f \chi_{[-T, T]}$ and
  $B_\Omega f = (\widehat{f} \chi_{[-\Omega, \Omega]})^\vee = f * d_{2\pi \Omega}$,
  then
  \[
    E_{T, \Omega}
    = \|A_T B_\Omega\|^2
    = \|B_\Omega A_T B_{\Omega}\|^2
    = \lambda_1,
  \]
  where $\lambda_1$ is the largest eigenvalue of $B_\Omega A_T B_\Omega$
  (recall that $B_\Omega A_T B_\Omega$ is compact and self-adjoint, so it has
  eigenvalues $\lambda_n \to 0$).
  The spectral theorem also gives
  orthonormal eigenvectors
  $\{\varphi_n\}_{n \in \N}$:
  \[
    B_\Omega A_T
    B_\Omega f
    = \sum_{n = 1}^\infty  \lambda_n \langle f, \varphi_n \rangle \varphi_n.
  \]
  Note that
  $B_\Omega (A_T B_\Omega \varphi_n) = \lambda_n \varphi_n$
  for $\lambda_n \ne 0$, so
  $\varphi_n \in \mathcal{F} L^2_{[-\Omega, \Omega]}(\R)$
  and $\supp(\widehat{\varphi}_n) \subseteq [-\Omega, \Omega]$.

  What is $\|A_T \varphi_n\|_2^2$?
  We can compute
  \begin{align*}
    \|A_T \varphi_n\|_2^2
    &= \langle A_T \varphi_n, A_T \varphi_n \rangle
    = \langle A_T B_\Omega \varphi_n, A_T B_\Omega \varphi_n \rangle \\
    &= \langle B_\Omega \underbrace{A_T A_T}_{A_T} B_\Omega \varphi_n, \varphi_n \rangle
    = \lambda_n \langle \varphi_n, \varphi_n \rangle
    = \lambda_n \|\varphi_n\|_2^2
    = \lambda_n.
  \end{align*}
  Thus we see that
  $E_{T, \Omega} = \lambda_1 = \|A_T \varphi_1\|_2^2 < \|\varphi_1\|_2^2 = 1$, so
  $\varphi_1$ has the greatest energy in
  $[-T, T]$.

  Note that $\{\varphi_n\}_{n \in \N}$
  is an orthonormal basis
  for its closed span, which a priori
  lies in $\mathcal{F} L^2_{[-\Omega, \Omega]}(\R)$.
\end{remark}

\begin{prop}\label{prop:onb-bandlimited}
  $\cspan\{\varphi_n\} = \mathcal{F} L^2_{[-\Omega, \Omega]}(\R)$.
\end{prop}

\begin{proof}
  We show $\cspan\{\varphi_n\}^\perp = \{0\}$. Suppose $f \in \mathcal{F} L^2_{[-\Omega, \Omega]}(\R)$ with $f \perp \varphi_n$
  for all $n$. Then
  \[
    f \in \cspan\{\varphi_n\}^\perp
    = \crange(B_\Omega A_T B_\Omega)^\perp
    = \ker(B_\Omega A_T B_\Omega).
  \]
  Thus we have
  \[
    \|A_T f\|_2^2
    = \langle A_T B_\Omega f, A_T B_\Omega f \rangle
    = \langle B_\Omega A_T B_\Omega f, f \rangle
    = 0,
  \]
  so $f = 0$ a.e. on $[-T, T]$. By
  the Paley-Wiener theorem, this implies
  $f = 0$.
\end{proof}

\begin{remark}
  Proposition \ref{prop:onb-bandlimited}
  implies that there are infinitely
  many nonzero eigenvalues
  for $B_\Omega A_T B_\Omega$.

  Moreover, since $B_\Omega$ is the
  orthogonal projection onto
  $\mathcal{F} L^2_{[-\Omega, \Omega]}(\R)$
  and $\{\varphi_n\}$ is an
  orthonormal basis for
  $\mathcal{F} L^2_{[-\Omega, \Omega]}(\R)$,
  we can write
  \[
    B_\Omega f
    = \sum_{n = 1}^\infty \langle f, \varphi_n \rangle \varphi_n.
  \]
  Note that $L^2_{[-T, T]}(\R)$
  is not orthogonal to
  $\mathcal{F} L^2_{[-\Omega, \Omega]}(\R)$:
  For example,
  \[
    \langle \chi_{[-T, T]}, d_{2\pi \Omega} \rangle
    = \int_{-T}^T \frac{\sin 2 \pi \Omega x}{\pi x} \, dx
    \ne 0.
  \]
  However, set
  $\psi_n = \lambda_n^{-1 / 2} A_T \varphi_n$.
  Then
  \[
    \langle \lambda_m^{-1 / 2} A_T \varphi_m, \lambda_n^{-1 / 2}A_T \varphi_n\rangle
    = \lambda_m^{-1 / 2} \lambda_n^{-1 / 2}
    \langle B_\Omega A_T B_\Omega \varphi_m, \varphi_n \rangle
    = \lambda_m^{1 / 2} \lambda_n^{-1 / 2}
    \langle \varphi_m, \varphi_n \rangle
    = \delta_{m n}.
  \]
  So $\{\psi_n\}_{n \in \N}$ is an
  orthonormal sequence in
  $L^2_{[-T, T]}(\R)$.
\end{remark}

\begin{exercise}
  Show that $\{\psi_n\}_{n \in \N}$
  is an orthonormal basis for
  $L^2_{[-T, T]}(\R)$.
\end{exercise}

\begin{remark}
  Note that
  \[
    \langle \varphi_m, \varphi_n \rangle
    = \int_{-\infty}^\infty \varphi_m(x) \overline{\varphi_n(x)} \, dx
    = 0, \quad m \ne n.
  \]
  Furthermore,
  \[
    \langle \psi_m, \psi_n \rangle
    = \lambda^{-1 / 2}_m \lambda^{-1 / 2}_n
    \int_{-T}^T \varphi_m(x) \overline{\varphi_n(x)} \, dx
    = 0, \quad m \ne n
  \]
  even when the $\varphi_n$ are
  \emph{not} supported in $[-T, T]$.

  By symmetry (swapping the roles of
  $A_T$ and $B_\Omega$), one can also
  compute that
  \[
    A_T B_\Omega A_T f
    = \sum_{n = 1}^\infty \lambda_n \langle f, \psi_n \rangle \psi_n.
  \]
  In particular,
  $A_T B_\Omega A_T$ and
  $B_\Omega A_T B_\Omega$
  have the same eigenvalues.
\end{remark}

\begin{prop}
  $B_\Omega A_T$ commutes with
  \[
    Kf
    = (T^2 - x^2) f''(x) - 2x f'(x) - 4\pi^2 \Omega^2 x^2 f(x).
  \]
\end{prop}

\begin{remark}
  Since $\varphi_n$ is already
  band-limited, we have
  \[
    B_\Omega A_T \varphi_n
    = B_\Omega A_T B_\Omega \varphi_n
    = \lambda_n \varphi_n.
  \]
  Moreover, Paley-Wiener
  implies $\varphi_n$ is infinitely
  differentiable, so
  $\varphi_n \in \domain(K)$. Then
  \[
    B_\Omega A_T K \varphi_n
    = K B_\Omega A_T \varphi_n
    = K(\lambda_n \varphi_n)
    = \lambda_n K \varphi_n.
  \]
  Note that $K \varphi_n \ne 0$, so
  $K \varphi_n$ is an eigenvector for
  $B_\Omega A_T$.
  The multiplicities of the $\lambda_n$
  is $1$, so $K \varphi_n = \mu \varphi_n$.
  The eigenfunctions $\varphi_n$
  of $K$ are known functions called the
  \emph{prolate spheroidal wave functions}.
\end{remark}

\section{Approximating Band-Limited Functions}

\begin{remark}
  We know that $1 > \lambda_1 > \lambda_2 > \cdots \to 0$.
  Also,
  \[
    \|A_T B_\Omega\|_{\mathrm{HS}}^2
    = \|g_{T, \Omega}\|_2^2
    = \int_{-T}^T \int_{-\infty}^\infty
    d_{2\pi \Omega}(x - y)^2 \, dx dy
    = 4T \Omega,
  \]
  where $g_{T, \Omega}$ is the
  corresponding kernel.
  The singular values of $A_T B_\Omega$ are
  \[
    s_n = \lambda_n ((A_T B_\Omega)^* (A_T B_\Omega))^{1 / 2}
    = \lambda_n (B_\Omega A_T B_\Omega)^{1 / 2}
    = \lambda_n^{1 / 2}.
  \]
  Thus $4T\Omega = \|A_T B_\Omega\|_{\mathrm{HS}}^2 = \sum_{n = 1}^\infty s_n^2 = \sum_{n = 1}^\infty \lambda_n$.
  On the other hand,
  $\|B_\Omega A_T B_\Omega\|_{\mathrm{HS}}^2 = \sum_{n = 1}^\infty \lambda_n^2$, and
  \[
    \|B_\Omega A_T B_\Omega\|_{\mathrm{HS}}^2
    = \|k_{T, \Omega}\|_2^2
    = \int_{-T}^T \int_{-T}^T
    d_{2\pi \Omega}(x - y)^2 \, dx dy
    < 4T \Omega.
  \]
  Due to the decay of
  $d_{2\pi \Omega}(x - y)^2$, for
  $T$ large, we should expect that
  $\|B_\Omega A_T B_\Omega\|_{\mathrm{HS}}^2 \to 4T\Omega$.
  Hence
  \[
    \sum_{n = 1}^\infty \lambda_n^2
    \approx \sum_{n = 1}^\infty \lambda_n.
  \]
  Thus intuitively, $\lambda_n$ should be
  either close to $1$ or close to $0$. In
  particular, only finitely many
  terms can be close to $1$, so
  \[
    f \approx \sum_{n = 1}^N \lambda \langle f, \varphi_n \rangle \varphi_n
    \in \mathrm{span}\{\varphi_1, \dots, \varphi_N\}.
  \]
  Thus the space of band-limited functions
  are approximately finite-dimensional
  in some sense.
\end{remark}

\begin{theorem}
  $\#\{n : \lambda_n \ge 1 - \epsilon\} \le 4 T \Omega - C_\epsilon \log (T\Omega)$.
\end{theorem}

\section{Space of Possible Time and Band Limits}

\begin{remark}
  Define the energies
  \begin{align*}
    \alpha = E_T(f) &= \|A_T f\|_2^2
    = \int_{-T}^T |f|^2 \\
    \beta = E_\Omega(f) &= \|B_\Omega f\|_2^2
    = \int_{-\Omega}^\Omega |\widehat{f}|^2.
  \end{align*}
  Which pairs $(\alpha, \beta)$ can be
  achieved? We have seen that
  $(\lambda_1, 1)$ and $(\lambda_2, 1)$
  are possible, while $(1, 1)$ is
  not by the Paley-Wiener theorem.
  Similarly,
  $(1, 0)$ and $(0, 1)$ are
  not possible by Paley-Wiener. Since
  the roles of $A_T$ and $B_\Omega$
  are symmetric, we get that
  $(1, \lambda_1)$ and
  $(1, \lambda_2)$ are possible.
  Note that if $(\alpha, 1)$ is
  achieved, then so is
  $(\alpha', 1)$ for any
  $\alpha' < \alpha$, and a similar
  statement holds for $(1, \beta)$.

  One can show the space of
  valid pairs is
  $(\alpha, \beta) \in [0, 1]^2$ such that
  \[
    \cos^{-1}\alpha + \cos^{-1} \beta
    \ge \cos^{-1}\lambda^{1 / 2},
  \]
  which is in the shape of an ellipse.
\end{remark}
